% von Wikipedia: partielle Funktion, vll. noch etwas abändern

\begin{definition}[partielle Funktion]
Der Begriff \concept{partielle Funktion} ist eine 
Verallgemeinerung des Begriffs \concept{Funktion}:
Während eine Funktion eine \pred{linkstotale} und \pred{rechtseindeutige}
Funktion muss eine partielle Funktion nur \pred{rechtseindeutig} sein.
Funktionen im ersten Sinn werden zur Abgrenzung von partiellen Funktionen
auch \concept{totale Funktionen} genannt.
Partielle Funktionen werden werden als Funktionen modelliert:

\begin{definition}[Partielle Funktion]
Eine partielle Funktion $f: X → Y$ kann durch die Funktion $f'$ wie folgt
modelliert werden:
$f': X → Y ∪ \set{⊥}$ mit
\begin{align*}
	f'(x) =
	\begin{cases}
		f(x), \text{falls} x ∈ Def(f) \\
		⊥, \text{sonst}
	\end{cases}
\end{align*}
mit $Y ∩ \set{⊥} = ∅$.
\end{definition}


