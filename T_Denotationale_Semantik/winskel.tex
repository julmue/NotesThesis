\documentclass{scrartcl}

\usepackage{header}

\title{Formale Semantik von Programmiersprachen}
\subtitle{Zusammenfassung von Winsekls Formal Semantics of Programming Languages}
\author{Julian Müller}

\begin{document}

\maketitle
\tableofcontents
\newpage

\section{Einleitung}

Die Semantik einer Sprache legt die Bedeutung syntaktisch wohlgeformter 
Ausdrücke und Sätze einer natürlichen oder formalen Sprache fest;
demnach legt die Semantik einer Programmiersprache die Bedeutung dieser 
formalen Sprache fest.

Eine besondere Form der Semantik ist die formale Semantik, die Gegenstand 
dieses Abschnitts sein wird: hier wird die Bedeutung einer Sprache durch die 
Interpretation in einem mathematischen Modell gegeben.
Dieser Ansatz eignet sich vor allem für formale Sprachen und ist insbesondere 
bei Programmiersprachen von großem Nutzen -- der Zweck dieses 
mathematischen Modells ist das Verständnis des Programmverhaltens:
\begin{itemize}
\item Die Suche eines Modells:
   Eine formale Semantik zu definieren macht viele wichtige subtile Eigenschaften 
   einer Programmiersprache explizit.
\item Ein vorhandenes Modell:
    Grundlage zur Verwendung mathematischer/logische Methoden bei der Programmanalyse:
    \begin{itemize}
    \item Laufzeitanalyse
    \item Verifikation
    \end{itemize}
\end{itemize}


\subsection{Einteilungen der Semantiken}

Es gibt verschiedene Arten von formalen Semantiken:
\begin{itemize}
\item Operationale Semantik (Interpretermethode):
	Operationale Semantik beschreibt die Bedeutung eines Programmes
	durch die Spezifikation seines Laufzeitverhaltens auf einer abstrakten Maschine.
	Die operationale Semantik ist eine syntaktische Spezifikation die durch 
	eine Auswertungsrelation über den Termen der Sprache gegeben ist.
\item Denotationale Semantik (mathemtische Semantik):
	Verwendet die folgenden mathematischen Konzepte:
	\begin{itemize}
	\item vollständige Halbordnungen (complete partial order/cpo).
	\item stetige Funktionen (continuous functions).
	\item kleinste Fixpunkte (least fixed points).
	\end{itemize}
\item Axiomatische Semantik:
	Legt die Bedeutung eines Programms durch die Anwendung von
	Beweisregeln in einer Programmlogik (Hoare-Kalkül) fest.
	Im Vordergrund der axiomatischen Semantik steht die Programmkorrektheit.
\end{itemize}

Diese drei Konzepte schließen sich gegenseitig nicht aus, sondern ergänzen sich
vielmehr. Jede hat ihre Stärken:
\begin{itemize}
\item Operationale Semantik ist wichtig für die Implementierung.
\item Axiomatische Semantik ist wichtig für Programmentwicklung und -verifikation.
\item Denotationale Semantik (super aber warum???).
\end{itemize}

Die unterschiedlichen Semantiken hängen auch voneinander ab:
Der Nachweis, dass die Beweisregeln einer axiomatischen Semantik korrekt sind
erfolgt relativ zur operationalen und denotationalen Semantik.
Der Nachweis, dass eine Implementierung korrekt ist 
erfolgt durch den Abgleich von operationaler und denotationaler Semantik.
Bei der Untersuchung der operationalen Semantik ist eine denotationale 
Semantik ein mächtiges Werkzeug:
Zum einen erlaubt sie die Abstraktion von unwichtigen Implementierungsdetails
zum anderen erlaubt sie die Anwendung des gesamten mathematischen 
Wissens und des deduktiven Insturmentariums der Interpretationsdomäne zur 
Untersuchung der Sprache und der in ihr geschriebenen Programme.

\subsection{Bedeutung der Logik in der Semantik}

Logik hat eine zentrale Rolle in der Semantik.

Gödels Unvollständigkeitssatz:
Unmöglichkeit einer vollständigen axiomatischen Semantik.


%-------------------------------------------------------------------------------
% \section{Einführung in die operationale Semantik}
\section{IMP -- eine einfache imperative Sprache}

Inhalt des Kapitels ist die Syntax und die Semantik(en) von \mobj{IMP}, einer 
minimalen, turing-vollständigen, prozeduralen Programmiersprache.

\subsection{Syntax von Imp}

In diesem Abschnitt wird die syntaktische Gestalt der Programme von \mobj{IMP}
festgelegt. Ein Programm ist hierbei synonym zu einem wohlgeformten Ausdruck
in \mobj{IMP} sowie zu einem Wort in der Sprache \mobj{IMP}.

\begin{definition}[Syntax von \mobj{IMP}]\label{def:syntaxIMP}
Die Menge der Ausdrücke \mobj{Exp} von \mobj{IMP} ist die Vereinigung 
der Ausdrucksmengen der syntaktischen Kategorien der Sprache:
\begin{itemize}
\item Ganzzahlen \mobj{N}: $\set{\dots, -2, -1, 0, 1, 2, \dots}$.
\item Wahrheitswerte \mobj{T}: $\set{true, false}$
\item Positionen \mobj{Loc}: $\comp{a^*}{a ∈ \set{a, \dots, z}}$.
\item \mobj{Aexp}: 
	$ a ::= n | X | a₀ \imp{+} a₁ | a₀ \imp{-} a₁ | a₀ \imp{*} a₁ $
\item \mobj{Bexp}: 
	$b ::= \imp{true} | \imp{false} | a₀ \imp{=} a₁ | a₀ \imp{<=} a₁ 
	 | \imp{not}b | b₀ \imp{&&} b₁ | b₀ \imp{||} b₁$
\item \mobj{Com}: 
	$c$ ::= $\imp{skip} | X \imp{:=} a | c₀\imp{;}c₁ | \imp{if} b \imp{then} c₀ \imp{else} c1 
	 | \imp{while} b \imp{do} c$
\end{itemize}
\end{definition}

\begin{remark}
Als \concept{Positionen} werden hier Speicherplätze bezeichnet die über einen 
bestimmten Namen adressiert werden können; 
Oft werden diese als (Programm-)Variablen bezeichnet ---
hier wird dieser Name bereits für ein anderes Konzept verwendet.
% welches?
\end{remark}

\begin{remark}
Für Definition~\ref{def:syntaxIMP} und im weiteren Verlauf gelten die folgenden Konvention:
\begin{itemize}
\item $n$, $m$ sind Metavariablen für Ganzzahlen \imp{N}.
\item $X$, $Y$ sind Metavariablen für Speicherplätzen \imp{LOC}.
\item $a$ sind Metavariablen für arithmetischen Ausdrücken \imp{Aexp}.
\item $b$ sind Metavariablen für boolschen Ausdrücken \imp{Bexp}.
\item $c$ sind Metavariablen für Kommandos \imp{Com}.
\end{itemize}
\end{remark}

\begin{definition}[Syntaktische Äquivalenz]
Syntaktische Äquivalenz $≡$ .
\end{definition}


%------------------------------------------------------------------------------
\subsection{Semantik \imp{Imp} I: Operationale Semantik}

Die Auswertung von Ausdrücken in der Sprache \mobj{Imp} kann durch die formale
Beschreibung einer abstrakten Maschine gegeben werden, welche in \mobj{Imp}
geschriebene Programme interpretiert und ihnen damit eine Bedeutung zuweist
(daher auch der Name (\concept{Interpreter})): 
Das operationale Verhalten dieser Maschine, also die Rechenoperationen zur
Auswertung der Programm, werden \concept{operationale Semantik} genannt.

Die operationale Semantik wird hier durch Transformationsregeln angegeben
welche die Auswertungsrelation/Evaluationsrelation $→ ⊆ \mobj{Exp} ×\mobj{Exp}$
definieren. 
% ist das nicht eher eine partielle Auswertungsfn?
Aus der Perspektive der operationalen Semantik ist die Bedeutung eines
Ausdrucks einer Sprache wiederum ein Ausdruck derselben Sprache.

Die zur Auswertung von \mobj{Imp} definierte Maschine hat einen unendlichen
Speicher für benannte Werte, die Programmvariablen (im weiteren Verlauf
Positionen genannt). Diese Speicherpositionen werden in Porgrammen direkt
reifiziert: Positionsnamen können definiert, Werte zugewiesen, überschrieben
und die mit einem Variablennamen assoziierten Werte ausgelesen werden.

\begin{definition}[Zustand]
Die Zustandsmenge $Σ$ besteht aus Funktionen $σ : Loc → N$ von Speicherplätzen
auf Zahlen (Momentaufnahem des Speichers);
$σ(X)$ ist damit der Wert, oder Inhalt, des Speicherplatzes $X$ im Zustand $σ$ .
\end{definition}

Die Auswertung eines Ausdrucks $e$ im Zustand $σ$
Repräsentation durch eine \concept{Konfiguration} repräsentiert. 

\begin{definition}[Konfigurationen]
Ein Tupel $k = <e,σ>$ mit $k ∈ Exp × Σ$ wird Konfiguration genannt.
\end{definition} 
% Stimmt so nicht ganz, Befehle sind da rausgenommen

Das Ergebnis der Auswertung eines Ausdrucks hängt damit von
zwei Komponenten ab: dem Ausdruck selbst und dem Zustand in dem sich die 
Maschine zu Beginn der Auswertung befindet.

\begin{definition}[Big Step Evaluation]
$$<c,σ> → σ'$$
\end{definition}

\begin{example}
$$<X:=5,σ> → σ'>$$
\end{example}

\newpage
\begin{definition}[Auswertungsrelation: Auswertung Ausdrücke]
Definition der syntax-gerichteten Auswertungsrelation $→ ⊆ \imp{Exp} × Σ$. %???

Arithmetische Ausdrücke:
\begin{itemize}
\item Zahlen $a = n$:
	\begin{prooftree*}
    \Infer0{(n,σ) → n}
	\end{prooftree*}
\item Speicherplätze $a = X$:
	\begin{prooftree*}
    \Infer0{(X,σ) → σ(X)}
	\end{prooftree*}
\item Summen $a = a₀ + a₁$:
	\begin{prooftree*}
	\Hypo{(a₀,σ) → n₀}
	\Hypo{(a₀,σ) → n₀}
	\Infer2{(a₀ + a₀, σ) → n}
	\end{prooftree*}
\item Differenzen $a = a₀ - a₁$:
	\begin{prooftree*}
    \Hypo{(a₀,σ) → n₀}    
    \Hypo{(a₁,σ) → n₁}
    \Infer2{(a₀ - a₁, σ) → n}
	\end{prooftree*}
\item Produkte $a = a₀ × a₁$:
	\begin{prooftree*}
    \Hypo{(a₀,σ) → n₀}   
    \Hypo{(a₁,σ) → n₁}
    \Infer2{(a₀ × a₁, σ) → n}
    \end{prooftree*}
\end{itemize}
Boolsche Ausdrücke:
\begin{itemize}
\item Wahrheitswerte $b ≡ b₀$:
    \[
	\begin{prooftree}
    \Infer0{(\imp{true},σ) → true}
	\end{prooftree}
	\begin{prooftree}
    \Infer0{(\imp{false},σ) → false}
	\end{prooftree}
    \]
\item Äquivalenz $b ≡ a₀ = a₁$:
	\begin{prooftree*}
	\Hypo{(a₀,σ) → n}
	\Hypo{(a₁,σ) → n}
	\Infer2[$a₀=a₁$]{(a₀ = a₁, σ) → \imp{true}}
	\end{prooftree*}
	\begin{prooftree*}
	\Hypo{(a₀,σ) → n}
	\Hypo{(a₁,σ) → n}
	\Infer2[$a₀≠a₁$]{(a₀ = a₁, σ) → \imp{false}}
	\end{prooftree*}
\item Vergleich $a ≡ a₀ ≤ a₁$:
	\begin{prooftree*}
    \Hypo{a₀,σ → n₀}    
    \Hypo{a₁,σ → n₁}
    \Infer2[$a₀ ≤ a₁$]{(a₀ - a₁, σ) → \imp{true}}
	\end{prooftree*}
	\begin{prooftree*}
    \Hypo{a₀,σ → n₀}    
    \Hypo{a₁,σ → n₁}
    \Infer2[$a₀ ≰ a₁$]{(a₀ - a₁, σ) → \imp{false}}
	\end{prooftree*}
\item Negation $b ≡ ¬b₀$:
    \[
	\begin{prooftree}
    \Hypo{(b₀,σ) → true}
    \Infer1{(¬b₀, σ) → false}
    \end{prooftree}
	\begin{prooftree}
    \Hypo{(b₀,σ) → true}
    \Infer1{(¬b₀, σ) → false}
    \end{prooftree}
    \]
\item Konjunktion $b ≡ b₀ ∧ b₁$
	\begin{prooftree*}
    \Hypo{a₀,σ → \imp{true}}    
    \Hypo{a₁,σ → \imp{true}}
    \Infer2{(a₀ ∧ a₁, σ) → \imp{true}}
	\end{prooftree*}
%	\begin{prooftree*}
%    \Hypo{a₀,σ → \imp{true}}    
%    \Hypo{a₁,σ → \imp{false}}
%    \Infer2{(a₀ ∧ a₁, σ) → \imp{false}}
%	\end{prooftree*}
%	\begin{prooftree*}
%    \Hypo{a₀,σ → \imp{true}}    
%    \Hypo{a₁,σ → \imp{true}}
%    \Infer2{(a₀ ∧ a₁, σ) → \imp{false}}
%	\end{prooftree*}
%	\begin{prooftree*}
%    \Hypo{a₀,σ → \imp{true}}    
%    \Hypo{a₁,σ → \imp{true}}
%    \Infer2{(a₀ ∧ a₁, σ) → \imp{false}}
%	\end{prooftree*}
\item Disjunktion $b ≡ b₀ ∨ b₁$
	\begin{prooftree*}
    \Hypo{a₀,σ → \imp{true}}    
    \Hypo{a₁,σ → \imp{true}}
    \Infer2{(a₀ ∨ a₁, σ) → \imp{true}}
	\end{prooftree*}
%	\begin{prooftree*}
%    \Hypo{a₀,σ → \imp{true}}    
%    \Hypo{a₁,σ → \imp{false}}
%    \Infer2{(a₀ ∨ a₁, σ) → \imp{true}}
%	\end{prooftree*}
%	\begin{prooftree*}
%    \Hypo{a₀,σ → \imp{true}}    
%    \Hypo{a₁,σ → \imp{true}}
%    \Infer2{(a₀ ∨ a₁, σ) → \imp{true}}
%	\end{prooftree*}
%	\begin{prooftree*}
%    \Hypo{a₀,σ → \imp{true}}    
%    \Hypo{a₁,σ → \imp{true}}
%    \Infer2{(a₀ ∨ a₁, σ) → \imp{false}}
%	\end{prooftree*}
\end{itemize}

\begin{itemize}
\item Atomare Befehle:
 	\begin{prooftree*}
 	\Infer0{(\imp{skip},σ) → σ}
 	\end{prooftree*}
	\begin{prooftree*}
	\Hypo{(a,σ) → m}
    \Infer1{(X:=a,σ) → σ[m/X]}
	\end{prooftree*}
\item Sequenzierung:
	\begin{prooftree*}
	\Hypo{(c₁,σ) → σ'}
	\Hypo{(c₀,σ) → σ''}
	\Infer2{(c₀;c₁,σ) → σ'}
	\end{prooftree*}
\item Konditionalse:
	\begin{prooftree*}
	\Hypo{(b,σ) → \imp{true}}
	\Hypo{(c₀,σ) → σ'}
	\Infer2{(\imp{if} b \imp{then} c₀ \imp{else} c₁,σ) → σ'}
	\end{prooftree*}
	\begin{prooftree*}
	\Hypo{(b,σ) → \imp{false}}
	\Hypo{(c₁,σ) → σ'}
	\Infer2{(\imp{if} b \imp{then} c₀ \imp{else} c₁,σ) → σ'}
	\end{prooftree*}
\item While-Schleifen:
	\begin{prooftree*}
	\Hypo{(b,σ) → false}
	\Infer1{(\imp{while} b \imp{do} c,σ) → σ}
	\end{prooftree*}
	\begin{prooftree*}
	\Hypo{(b,σ) → \imp{true}}
	\Hypo{(c,σ) → σ''}
	\Hypo{(\imp{while} b \imp{do} c,σ'') → σ'} 
	\Infer3{(\imp{while} b do c,σ) → σ'}
	\end{prooftree*}
\end{itemize}
\end{definition}

\begin{remark}[Initialzustand]
Zu Beginn der Programmausführung sind alle Register auf $0$ gesetzt:
$σ₀$ ist der initiale Zustand für den gilt: $∀ X ∈ \mobj{Loc}: σ₀(X)=0$.
\end{remark}

\begin{remark}[Konvergenz und Divergenz]
Die Ausführung eines Programms kann ein einem Finalzustand $σ'$ terminieren
(Konvergenz) oder nicht terminieren (Divergenz).
\end{remark}

\begin{remark}[Regelinstanz (Instanziierung einer Regel)]
Eine Regelinstanz entsteht durch einsetzen von Werten aus den entsprechenden
syntaktischen Kategorien und der Zustandsmenge in die Metavariablen. Beispiel:
\begin{prooftree*}
\Hypo{(2,σ₀) → 2}   
\Hypo{(3,σ₀) → 3}
\Infer2{(2 * 3, σ₀) → 6}
\end{prooftree*}
\end{remark}

\begin{remark}[Analyse der Operationen $+$, $-$, $×$]
Die Operationen $+$, $-$, $×$ werden an dieser Stelle nicht weiter analysiert;
die entsprechenden Reduktionen als gegeben vorausgesetzt.
\end{remark}


\begin{remark}[Derivationsbaum/Derivation]
Für ein Tripel $(a₀,σ) → a₁$ mit $a₀, a₁ ∈ \mobj{Exp}$ und $σ ∈ Σ$ ist die 
Konstruktion eines Derivationsbaums ein Entscheidungsverfahren ob es sicht
in der Auswertungsrelation befindet: 
Kann der Derivationsbaum abgeschlossen werden, bestehen seine Blätter also
nur noch aus Axiomen, ist das Tripel enthalten, sonst nicht.
% braucht es da nicht eigentlich mehr als ein Tripel
% * (a,σ) → (a,σ)	als wirkliche Auswertungsrelation
% * (a,σ) → a		als Auswertungsrelation wo der Endzustand nicht interessiert
% * a → a			als Auswertungsrelation ab dem leeren Startzustand
\begin{example}[Ableitungsbaum für $(\imp{(Init + 5) + (5+9)}, σ₀) → \imp{21}$]
\begin{prooftree*}
\Infer0{(9,σ₀) → 9}
\Infer0{(7,σ₀) → 7}
\Infer2{(7 + 9,σ₀) → 16}
\Infer0{(5,σ₀) → 5}
\Infer0{(Init,σ₀) → 0}
\Infer2{((Init + 5,σ₀) → 5)}
\Infer2{((Init + 5) + (5+9)), σ₀) → 21}
\end{prooftree*}
\end{example}
\end{remark}

\begin{remark}[Nichtdeterministische Evaluation]
Auch die Auswertung eines Ausdrucks kann durch einen
Ableitungsbaum/Derivationsbaum (kurz: Derivation) bestehend aus Regelinstanzen
gelöst werden. Hier müssen für das Antezedens der Konklusion strukturgleiche
Regeln gefunden und instanziiert werden.
Einige dieser Vorderglieder können Instanzen gleich mehrerer Regeln sein; Um in 
jedem Fall eine (abgeschlossene) Ableitung zu finden, falls diese existiert, 
kann es notwendig sein jede passende Regel zu instanziieren.

Damit sind die Regeln ein Algorithmus zur Auswertung von Ausrücken;
sie geben diesen Ausdrücken eine operationale Bedeutung.
Die hier vorgestellte Definition über Regeln, \concept{strukturelle} oder auch
\concept{natürliche Semantik} genannt ist nur eine der denkbaren 
Darstellungsformen für operationale Semantiken, wenn auch eine sehr verbreitete.
Auch andere Beschreibungen -- beispielsweise natürlichsprachliche --
sind möglich. Beispiel: Operationale Semantik von Java und C++.
\end{remark}


\begin{remark}[Semantische Äquivalenzrelation]
Über die Auswertungsrelation kann eine semantische Äquivalenzrelation definiert
werden:
Zwei Ausdrücke/Programme sind semantisch äquivalent, wenn sie in allen Zuständen
Wertverlaufsgleich sind modulo der Auswertungsrelation.

$$a₀ \sim a₁ \text{gdw.} (∀ n ∈ ℕ, σ ∈ Σ: (e₀,σ) → n ⇔ (e₁,σ) → n) $$
$$b₀ \sim b₁ \text{gdw.} (∀ b ∈ 𝔹, σ ∈ Σ: (b₀,σ) → n ⇔ (b₁,σ) → n) $$
$$c₀ \sim c₁ \text{gdw.} (∀ σ, σ' ∈ Σ: (c₀,σ) → n ⇔ (c₁,σ) → n) $$

\end{remark}

\begin{todo}
Was nocht aufzudröseln ist: Smallstep vs Bigstep relation und über was genau
diese Relationen jetzt eigentlich definiert sind...
Außerdem trennt er nochmal Evaluation und Exekution
\end{todo}


%-------------------------------------------------------------------------------
%-------------------------------------------------------------------------------
\section{Induktive Definitionen}

Theorie der induktiv definierten Mengen,
von welchen die syntax und operativen Semantiken Beispiele sind.
Mengen, welche durch Regeln definiert werden sind die kleinsten (least)
Mengen die abgeschlossen sind unter der Regelanwendung.

\subsection{Regelinduktion}

\subsection{Spezielle Regelinduktion}

\subsection{Beweisregeln für die operationale Semantik}

\subsection{Operatoren und ihre kleinsten Fixpunkte}


%-------------------------------------------------------------------------------
%-------------------------------------------------------------------------------
\section{Denotationale Semantik von \mobj{IMP}}

% Motivation
Denotation of a command is a partial function of states.
Denotationalle Semantiken lassen sich für fast alle Programmiersprachen definieren,
allerdings gibt es Probleme bei Parallelismus und 'fairness'.

\begin{itemize}
\item Ein arithmetischer Ausdruck $a ∈ \mobj{Aexp}$ bezeichnet eine Funktion $𝓐⟦a⟧:Σ → N$.
\item Ein boolscher Ausdruck $a ∈ \mobh{Bexp}$ bezeichnet eine Funktion $𝓑⟦b⟧:Σ → 𝔹$.
\item Ein Befehl $c$ bezeichnet eine partielle Funktion $𝓒⟦c⟧:Σ → Σ$.
\end{itemize}

\begin{remark}
Die \concept{Oxford brackets} genannten Klammersymbole $⟦$ und $⟧$ sind
traditionell in der denotationalen Semantik.
Sie sollen anzeigen, dass es sich bei dem Argument um einen syntaktischen
Ausdruck handelt (der an dieser Stelle nicht ausgewertet wird).
\end{remark}

Die Semantischen Funktionen 
\begin{itemize}
\item $𝓐: \mobj{Aexp} → (Σ → N)$
\item $𝓑: \mobj{Bexp} → (Σ → T)$
\item $𝓒: \mobj{Com} → {Σ → Σ}$
\end{itemize}
werden über strukturelle Induktion definiert.


\end{document}
