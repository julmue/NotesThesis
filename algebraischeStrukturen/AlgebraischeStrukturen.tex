\section{Algebraische Strukturen}

[Wiki](https://de.wikipedia.org/wiki/Mathematische_Struktur)
Basieux
Bourbaki

\begin{definition}[mathematische Struktur]

Eine mathematische Struktur ist eine Menge (mit Relationen über dieser Menge) 
wo sowohl Menge als auch Relationen bestimmte Eigenschaften aufweisen.

Diese Eigenschaften ergeben sich durch eine oder mehrere Relationen zwischen:
\begin{itemize}
\item den Elementen dieser Menge (Struktur erster Stufe)
\item Teilmengen der Menge (Struktur zweiter Stufe)
\end{itemize}

Der \emph{Typ} einer Struktur wird durch \emph{Axiome} festgelegt;
eine Menge ist Instanz eines bestimmten Strukturtyps, wenn ihre Relationen
die Axiome des Typs erfüllen. (Modelltheorie?)

Klassifikation der wichtigsten Strukturen (Bourbaki):
\begin{itemize)
\item algebraische Strukturen
\item Ordnungsstrukturen
\item topologische Strukturen
\end{itemize}
(manchmal noch geometrische Strukturen)

\todo{Motivation Basieux}

Eine Menge kann, auch klassenübergreifend, Instanz mehrere Strukturtypen sein.
\todo{Terminologie so richtig?}



%-------------------------------------------------------------------------------
%-------------------------------------------------------------------------------
\subsection{Algebraische Strukturen}

Eine algebraische Struktur (kurz: Algebra) ist eine Struktur (erster Stufe),
die nur durch eine oder mehrere Verknüpfungen definiert ist.
(Verknüpfungen = Funktionen = spezielle Relationen)



%-------------------------------------------------------------------------------
\subsubsection{Strukturen mit einer inner Verknüpfungen: Gruppen und ähnliche}

Die fundamentalen algebraischen Strukturen besitzen ein oder zwei 
zweistellige innere Verknüpfungen (Operationen).

Die Klassifikation wird anhand der folgenden Axiomen vorgenommen:
\begin{itemize}
\item (E)  Abgeschlossenheit: 			$∀ a, b ∈ M: a ∘ b ∈ M$
\item (A)  Assoziativität: 				$∀ a, b, c ∈ M: (a ∘ b) ∘ c = a ∘ (b ∘ c)$
\item (N)  Existenz eines neutralen Elements: 	$∃ e ∈ M: ∀ a ∈ M: a ∘ e = e ∘ a = a$
\item (I)  Existenz eines inversen Elements: 	$∀ a ∈ M: ∃ a^{-1} ∈ M: a ∘ a^{-1} = a^{-1} ∘ a = e$
\item (K)  Kommutativgesetz: 			$∀ a, b ∈ M: a ∘ b = b ∘ a$
\item (Ip) Idempotenzgesetz: 			$∀ a ∈ M: a ∘ a = a$
\end{itemize}

die folgenden Strukturen mit einer zweistelligen inneren Verknüpfung 
verallgemeinern oder spezialisieren den fundamentalen Begriff der \emp{Gruppe}:

\todo{Liste der Strukturen}



%-------------------------------------------------------------------------------
\subsubsection{Strukturen mit zwei inneren Verknüpfungen: Ringe, Körper und ähnliche}

Die folgenden Strukturen haben zwei innere Verküpfungen, die gewöhnlich 
als \emph{Addition} und \emph{Multiplikation} geschrieben werden;
Diese Strukturen sind von den Zahlenbereichen $ℤ$, $ℚ$ und $ℝ$ abstrahiert,
mit denen man gewöhlich rechnet.

Die Verträglichkeit der additiven und multiplikativen Verknüpfung wird durch
folgende Axiome sichergestellt:
\begin{itemize}
\item (Dl) Links-Distributivgesetz: $∀ a, b, c ∈ M: a * (b + c) = a * b + a * c$
\item (Dr) Rechts-Distrubutivgesetz: $∀ a, b, c ∈ M: (a + b) * c = a * c + b * c$
\item (D)  Distrubutivgesetz: es gelten Dl und Dr.
\end{itemize}

Weitere Axiome, die beide Verknüpfungen betreffen sind:
\begin{itemize}
\item (U) Ungleichheit: $1 ≠ 0$ 
	(die neutralen Elemente der additiven und multiplikativen Operation sind ungleich)
\item (T) Nullteilerfreiheit:
	Wenn $0$ das neutrale Element der additiven Verknüpfung bezeichnet,
	dann folgt aus $a * b = 0$ für alle $a, b ∈ M$, dass $a = 0$ oder $b = 0$ gilt.
\item (I*) Inverses*:
	$∀ a ∈ M\\\{0\}: ∃a^{-1} ∈ M: a * a^{-1} = a^{-1}a * a = e$
	Für jedes Element, mit Ausnahme des neutralen Elements der additiven Verknüpfung,
	existiert eine inverses Element bezüglich der multiplikativen Verknüpfung.
\end{itemize}

\todo{Liste der Strukturen}




%-------------------------------------------------------------------------------
\subsubsection{Strukturen mit zwei Verknüpfungen: Verbände, Mengenalgebren und ähnliche}

Ein \emph{Verband} ist eine algebraische Struktur, dessen zwei inneren Verknüpfungen
im Allgemeinen nicht als Addition und Multiplikation aufgefasst werden können.

Folgendes Axiom ist relevant:
\begin{itemize}
\item Verschmelzung (Absorption): $a ∩ (a ∪ b) = a$
\item Verschmelzung (Absorption): $a ∪ (a ∩ b) = a$
\end{itemize}

\todo{Liste der Strukturen: Verband, distributiver Verband}



%-------------------------------------------------------------------------------
\subsubsection{Strukturen mit inneren und äußeren Verknüpfungen: Vektorräume und ähnliche}

\todo{ergänzen}


\begin{comment}
%-------------------------------------------------------------------------------
\subsection{Zusätzliche algebraische Strukturen auf Vektorräumen}

\todo{ergänzen}
\end{comment}






%-------------------------------------------------------------------------------
%-------------------------------------------------------------------------------
\subsection{Ordnungsstrukturen}

Eine Ordnungsstruktur ist eine Struktur (erster Stufe), die mit einer \emph{Ordnungsrelation}
ausgestattet ist (relationale Struktur/Relativ).

\todo{ergänzen}
% Quasiordnung: reflexiv und transitiv
% Teilordnung: (partielle Ordnung, Halbordnung) reflexiv, antisymmetrisch, transitv
% strenge Halbordung: irreflexiv und transitiv
% totale Ordnung (lineare Ordnung): totale Halbordnung
% strenge Totalordnung: total, irreflexiv, transitiv
% fundierte Ordnung
% Wohlordnung



%-------------------------------------------------------------------------------
%-------------------------------------------------------------------------------
\subsection{Ordnungsstrukturen}

Die generalisierung des geometrische Begriffs des \emph{Abstands} (der Metrik)
ermöglicht es, in \emph{metrischen Räumen} das grundlegende Konzept der modernen
Analysis, die \emph{Konvergenz} zu handhaben.

\emph{Topologische Räume} sind aus dem Bemühen hervorgeganen,
die Konvergenz in einem allgemeinen Sinn zu behandeln
(jeder Metrische Raum ist ein topologischer Raum mit der Topologie die durch 
die Metrik induziert wird.)


