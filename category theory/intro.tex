\documentclass[parskip=half]{scrreprt}
\usepackage{common}

%-------------------------------------------------------------------------------
% map			Abbildung
% morphism		Abbildung
% domain		Wertemenge
% codomain		Zielmenge
% rule			Abbildungsvorschrift
% process		Abbildungsvorschrift
% isomorphism	Isomorphismus

%-------------------------------------------------------------------------------
\begin{document}
\title{Zusammenfassung: Einführung in die Kategorientheorie}
\subtitle{Zusammenfassung \enquote{Introduction to Category Theory}}
\author{Julian Müller}
\maketitle
\tableofcontents

\part{Kategorien}

\chapter{Definition der Kategorie}

\begin{definition}[Kategorie]
Eine Kategrie $𝓒$ besteht aus:
\begin{itemize}[leftmargin=1em]
\item Daten: 
	\begin{itemize}
	\item Ob($𝓒$): Eine Klasse von Objekten ($A, B, C, \dots$).
	\item Ar($𝓒$): Eine Klasse von Pfeilen/Morphismen ($f, g, h, \dots$). \\
		Jeder Morphismus $f : A → B$ besteht aus:
		\begin{itemize}
		\item Quelle/Domäne: $dom(f) = A$
		\item Ziel/Kodomäne: $codom(f) = B$
		\end{itemize}
	\item Verknüpfungsabbildung für je zwei Abbildungen:
		$$(B → C) × (A → B) ⇒ (A → C), (g,f) ↦ g ∘ f$$
	\item Identitätsmorphismus: $1_A : A → A$ (auch $id_X$) für jedes Objekt.
	\end{itemize}
\item Gesetzen:
	\begin{itemize}
	\item Identitätsgesetze:
		\begin{itemize}
		\item Wenn $A \xrightarrow{1_A} A \xrightarrow{g} B$, dann $g ∘ 1_A = g$.
		\item Wenn $A \xrightarrow{f} B \xrightarrow{1_B} B$, dann $1_B ∘ f = f$.
		\end{itemize}
	\item Assoziativgesetz:
		\begin{itemize}
		\item Wenn $A \xrightarrow{f} B \xrightarrow{g} C \xrightarrow{h} D$\\ 
			dann $A \xrightarrow{h∘(g∘f)=(h∘g)∘f} D $
		\end{itemize}
	\end{itemize}
\end{itemize}
\end{definition}

\begin{remark}
Die Algebra der Abbildungskomposition 
(Komposition als die Verküpfung von Abbildungen)
gleicht der Multiplikation von Zahlen,
allerdings hat sie eine weit reichere Interpretation.
\end{remark}

\begin{remark}
Die Identitätsabbildung ist das neutrale Element der Abbildungsalgebra,
wie die $1$ das neutrale Element der Multiplikation ist ... daher auch $1_A$.
\end{remark}

\begin{remark}
Der Kompositionsoperator $∘$ verküpft zwei Abbildungen $g$ und $h$ zu 
einer neuen Abbildungen $h∘g$ ($h$ nach $g$).
\end{remark}

\chapter{Die Kategorie der Mengen}

Ein Beispiel für eine Kategorie:
Die Kategorie der endlichen Mengen, und Abbildungen.
Eine Abbildung \engl{Abbildung}{map} ist eine Prozess 
um von einer Menge in eine andere zu kommen.

% Nach Meijer also unsere erste Programmiersprache
% endliche Mengen als Typen
% Abbildungen als Funktionen
\begin{definition}[Kategorie der endlichen Mengen]
\begin{description}
\item[Objekte] finite Mengen $A, B, \dots$
\item[Morphismen] Abbildungen zwischen Mengen $f: A → B$
	\begin{description}
	\item[Definitionsmenge] eine Menge $A$ \engl{Definitionsmenge}{domain}
	\item[Zielmenge] eine Menge $B$ \engl{Zielmenge}{codomain}
	\item[Abbildungsvorschrift] Zu jedem $a ∈ A$ ein $b ∈ B$ mit $f(a) = b$.
	\end{description}
\end{description}
\end{definition}

\begin{remark}
Die Abbildung ist linkstotal und rechtseindeutig
\end{remark}

Finite Mengen sind die Objekte der Kategorie der finiten Mengen.
Der Inhalt von Mengen lässt sich in verschiedenen Auflösungsgraden
als internes Diagramm darstellen.

\graphics{pics/internalDiagram1.png}{Internes Diagramm einer Menge}

Abbildungen zwischen Mengen sind die Morphismen zwischen den Objekten
der Kategorie der finiten Mengen.
Die Abbildungsvorschrift kann explizit gemacht werden durch ein internes 
Diagramm. 

\graphics{pics/internalDiagram2.png}{Internes Diagramm einer Abbildung}

Externe Diagramme zeigen nur Definitions- und Zielmenge; 
sie werden verwendet wenn es auf die genaue Abbildungsvorschrift nicht 
ankommt, oder wenn Morphismen zwischen mehreren Objekten untersucht werden.
\graphics{pics/externalDiagram1.png}{Externe Diagramme}

Endoabbildungen sind spezielle Abbildungen.
\begin{definition}[Endoabbildung]
Eine Endoabbildung \engl{Endoabbildung}{endomap} $f: A → A$ hat den
die selbe Definitions und Zielmenge.
\end{definition}

Identitätsabbildungen sind spezielle Endoabbildungen.

\begin{definition}[Identitätsabbildung]
Eine spezielle Endoabbildung ist die Identitätsabbildung $1_A$,
\engl{Identitätsabbildung}{identity map} die jedem Element von $A$ sich selber 
zuordnet.
\end{definition}

Die Verknüpfungsabbildung zweier Morphismen $f$ und $g$ in der Kategorie der 
finiten Mengen ist einfach die Komposition $g∘f$ der Abbildungen;
Voraussetzung für die Komposition ist, dass die Zielmenge von $f$ identisch
mit der Definitionsmenge von $g$ ist.

\graphics{pics/functionComposition1.png}{Funktionskomposition}

\begin{definition}[Punkt]% Unit?
Ein Punkt (auch Singleton) ist eine Menge mit exakt einem Element.
\end{definition}

\begin{remark}
Einelementige Mengen werden auch mit $1$ bezeichnet.
\end{remark}

\begin{theorem}
Die Komposition eines Punktes mit einer anderen Abbildung ist wieder ein Punkt.
\end{theorem}


%%%%%%%%%%%%%%%%%%%%%%%%%%%%%%%%%%%%%%%%%%%%%%%%%%%%%%%%%%%%%%%%%%%%%%%%%%%%%%%%
%% Session 2: Mengen, Abbildungen und Komposition                             %%
%%%%%%%%%%%%%%%%%%%%%%%%%%%%%%%%%%%%%%%%%%%%%%%%%%%%%%%%%%%%%%%%%%%%%%%%%%%%%%%%

Vergleich/Identität von Abbildungen;
Abbildungsvorschrift ist der Inhalt, Abbildung ist der Umfang.

\begin{example}
$$f : ℕ → ℕ, x ↦ (x+1)²$$
$$g : ℕ → ℕ, x ↦ x²+2x+1$$

Beide Abbildungen sind wertverlaufsgleich --- es handelt sich um die selbe
Abbildung $f=g$. Das ist exakt die Aussage der folgenden Gleichung:

$$ (x+1)² = x²+2x+1$$

Im Allgemeinen drücken Gleichungen aus, das Ausdrücke mit (evtl. verschiedenen)
Inhalten den gleichen Umfang haben.
Der Begriff der Abbildung wird damit extensional verwendet,
da eine Abbildung von ihrem Umfang bestimmt wird, nicht intensional von  
ihrere Abbildungsvorschrift.
\end{example}

\begin{remark}
Eine notwendige, aber nicht hinreichende Bedingung für die identität zweier 
Abbildungen $f$ und $g$ ist in jeder Kategorie, dass $f$ und $g$ die gleichen 
Domänen und Kodomänen haben.
\end{remark}

\begin{remark}[Test der Identität zweier Abbildungen]
Um die Identität zweier Abbildungen $A \xrightarrow{f} B$ und $A \xrightarrow{g} B$
zu beweisen genügt es zu zeigen:
$$\text{Wenn für jeden Punkt} 1 \xrightarrow{a} A, f ∘ a = g ∘ a, \text{dann} f=g$$
\end{remark}

% Vielleicht als Beispiel: Fahrenheit nach Celsius S. 28

\begin{remark}[Unterschied zwischen Multiplikation und Komposition]
Multiplikation auf $ℕ$ ist kommutativ, Komposition nicht.
\end{remark}


\begin{remark}[Anzahl der Abbildungen von einer Menge in eine andere Menge]
Für endliche Mengen $A$ und $B$ gibt es $|B|^{|A|}$ viele Abbildungen 
$f : A → B$.
\end{remark}


%%%%%%%%%%%%%%%%%%%%%%%%%%%%%%%%%%%%%%%%%%%%%%%%%%%%%%%%%%%%%%%%%%%%%%%%%%%%%%%%
%% Session 3: Komposition von Abbildungen und Zählen von Abbildungen		  %%
%%%%%%%%%%%%%%%%%%%%%%%%%%%%%%%%%%%%%%%%%%%%%%%%%%%%%%%%%%%%%%%%%%%%%%%%%%%%%%%%

\begin{remark}
Die Menge der Abbildungen $f : A → B$ wird mit $B^A$ bezeichnet
(Da sie $|B|^{|A|}$ viele Elemente enthält).
\end{remark}


%%%%%%%%%%%%%%%%%%%%%%%%%%%%%%%%%%%%%%%%%%%%%%%%%%%%%%%%%%%%%%%%%%%%%%%%%%%%%%%%
%%%%%%%%%%%%%%%%%%%%%%%%%%%%%%%%%%%%%%%%%%%%%%%%%%%%%%%%%%%%%%%%%%%%%%%%%%%%%%%%
\setpartpreamble{%
\begin{abstract}
Wir untersuchen folgende Analogie: Wenn die Komposition von Abbildungen
der Multiplikation von Zahlen gleicht, was ist dann das Analogon zur Division?
\end{abstract}
}
\part{Die Algebra der Komposition}


%%%%%%%%%%%%%%%%%%%%%%%%%%%%%%%%%%%%%%%%%%%%%%%%%%%%%%%%%%%%%%%%%%%%%%%%%%%%%%%%
%% Artikel 2: Isomorphismen, 
%%%%%%%%%%%%%%%%%%%%%%%%%%%%%%%%%%%%%%%%%%%%%%%%%%%%%%%%%%%%%%%%%%%%%%%%%%%%%%%%
\chapter{Isomorphismen, Sektionen, Retraktionen, Idempotenz, Automorphismen}

\subsection{Isomorphismen}

\begin{example}[Einführung: Vergleich von Mengen]
Isomorphismen erlauben unter abderem den Vergleich von Mengen ohne einen Zahlbegriff.
Ein Isomorphismus eine spezielle Abbildung. Welche Eigenschaften muss eine 
Abbildung $f: A → B$ haben, damit man zwei mit ihrer Hilfe zwei Mengen $A$ und 
$B$ vergleichen kann? Sie muss 
\begin{enumerate*}
\item injektiv und 
\item surjektiv 
\end{enumerate*}
sein --- eine Bijektion. Dann gibt es eine Umkehroperation $f^{-1}: B → A$ 
mit $f ∘ f^{-1} = 1_A$ und $f^{-1} ∘ f = 1_B$.
Beide dieser Gleichungen müssen erfüllt sein, damit $A$ und $B$ gleich mächtig
sind.
\end{example}

\graphics{pics/setIsomorphism1}{Isomorphismus zwischen Mengen}
% ein isomorphismus ist nur zwischen Mengen mit der gleichen Kardinalität möglich.
% ist die Anzahl der Isomorphismen zwischen zwei Mengen gleich ihrer Kardinalität?

\begin{definition}[Isomorphismus]
Ein Morphismus $A \xrightarrow{f} B$ wird \definiendum{Isomorphismus} 
(oder umkehrbare Abbildung) genannt, wenn es eine Abbildung $B \xrightarrow{g} A$ 
gibt für welche die folgenden Gleichungen gelten:
\begin{enumerate}
\item $f ∘ g = 1_A$
\item $g ∘ f = 1_B$
\end{enumerate}
Eine Abbildung $g$ welche in Bezug auf $f$ diese Eigenschaften erfüllt heißt 
\concept{Inverse von $f$}. Zwei Objekte $A$ und $B$ werden isomorph genannt,
wenn es mindestens einen Isomorphismus $A \xrightarrow{f} B$ gibt.
\end{definition}

\begin{theorem}[Isomorphie ist eine Äquivlanzrelation]
Isomorphie zwischen zwei Objekten ist eine Äquivalenzrelation. Es gilt:
\begin{enumerate}
\item \label{def:IsoRef} Reflexivität: $A$ ist isomorph zu $A$
\item \label{def:IsoSym} Symmetrie: Wenn $A$ isomorph ist zu $B$, 
							dann ist $B$ isomorph zu $A$.
\item \label{def:IsoTra} Transitivität: Wenn $A$ isomorph zu $B$, 
							und $B$ isomorph zu $C$,
							dann ist $A$ isomorph zu $C$.
\end{enumerate}
\end{theorem}

\begin{proof}
Seien $A$, $B$ und $C$ Objekte und $f: A → B$, $f^{-1}: B → A$ sowie $g: B → C$ 
und $g^{-1}: C → B$ Isomorphismen (mit ihren Umkehrabbildungen).
% \begin{enumerate} 
% \item $f ∘ f^{-1} = 1_A$ und
% \item $f^{-1} ∘ f = 1_B$ 
% \item $g ∘ g^{-1} = 1_B$ und
% \item $g^{-1} ∘ g = 1_C$ 
% \end{enumerate}
Dann gilt:

\begin{subproof}[Reflexivität]
Aufgrund der Identitätsabbildung $1_A$.
\end{subproof}
\begin{subproof}[Symmetrie] 
Aufgund der Voraussetzung.
\end{subproof}
\begin{subproof}[Transitivität]
\begin{align*}\label{sp:IsoTrans}
	  &f ∘ g ∘ g^{-1} ∘ f^{-1}  \\	% Assoziativität
	= &f ∘ 1_B ∘ f^{-1}			\\% Annahme
	= &f ∘ f^{-1}				\\% Identitätsgesetz
	= &1_A						% Annahme
\end{align*}
Weiter $ g ∘ f ∘ f^{-1}∘ g^{-1} = 1_C$ analog.
\end{subproof}
\end{proof}







\end{document}

