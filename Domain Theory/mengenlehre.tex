\begin{comment}
<!-- ----------------------------------------------------------------------- -->
## Mengenlehre

### Mengen

#### Fundierungsaxiom
Fundierungsaxiom (axiom of foundation) auch Regularitätsaxiom (axiom of regularity):
Axiom von ZF.

Aussage: 
Elementketten (Ketten aus Elementbeziehungnen) sind immer endlich;
daraus folgt: Keine Menge kann Element von sich selbst sein.

Verhindert Russells Paradoxon.


### Funktionen


#### partielle Funktionen
partielle Funktionen:
* definiert
* undefiniert


#### Lambdanotation (anonyme Funktionen)
Praktisch: intensionale Funktionsbeschreibung (Beschreibung der Abbildungsvorschriften)
durch Verwendung von Lambdanotation.

Vorteil: Anonyme Funktionen

#### Funktionscomposition


### Abzählbare Menge
Jede Menge mit einer Bijektion mit einer Untermenge der natürlichen Zahlen 
heißt abzählbar.
	

### Cantors Diagonalargument

Diagonalbeweise (als Beweismethode) sind in der Informatik häufig anzutreffen.
\end{comment}
