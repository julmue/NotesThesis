\documentclass[parskip=half]{scrreprt}
\usepackage{common}

%-------------------------------------------------------------------------------
% map			Abbildung
% morphism		Abbildung
% domain		Wertemenge
% codomain		Zielmenge
% rule			Abbildungsvorschrift
% process		Abbildungsvorschrift
% isomorphism	Isomorphismus

%-------------------------------------------------------------------------------
\begin{document}
\title{Zusammenfassung: Einführung in die Kategorientheorie}
\subtitle{Zusammenfassung \enquote{Introduction to Category Theory}}
\author{Julian Müller}
\maketitle
\tableofcontents

\part{Kategorien}

\chapter{Definition der Kategorie}

\begin{definition}[Kategorie]
Eine Kategrie $𝓒$ besteht aus:
\begin{itemize}[leftmargin=1em]
\item Daten: 
	\begin{itemize}
	\item Ob($𝓒$): Eine Klasse von Objekten ($A, B, C, \dots$).
	\item Ar($𝓒$): Eine Klasse von Pfeilen/Morphismen ($f, g, h, \dots$). \\
		Jeder Morphismus $f : A → B$ besteht aus:
		\begin{itemize}
		\item Quelle/Domäne: $dom(f) = A$
		\item Ziel/Kodomäne: $codom(f) = B$
		\end{itemize}
	\item Verknüpfungsabbildung für je zwei Abbildungen:
		$$(B → C) × (A → B) ⇒ (A → C), (g,f) ↦ g ∘ f$$
	\item Identitätsmorphismus: $1_A : A → A$ (auch $id_X$) für jedes Objekt.
	\end{itemize}
\item Gesetzen denen die Morphismen genügen müssen:
	\begin{itemize}
	\item Identitätsgesetze:
		\begin{itemize}
		\item Wenn $A \xrightarrow{1_A} A \xrightarrow{g} B$, dann $g ∘ 1_A = g$.
		\item Wenn $A \xrightarrow{f} B \xrightarrow{1_B} B$, dann $1_B ∘ f = f$.
		\end{itemize}
	\item Assoziativgesetz:
		\begin{itemize}
		\item Wenn $A \xrightarrow{f} B \xrightarrow{g} C \xrightarrow{h} D$\\ 
			dann $A \xrightarrow{h∘(g∘f)=(h∘g)∘f} D $
		\end{itemize}
	\end{itemize}
\end{itemize}
\end{definition}

\begin{remark}
Die Algebra der Abbildungskomposition 
(Komposition als die Verküpfung von Abbildungen)
gleicht der Multiplikation von Zahlen,
allerdings hat sie eine weit reichere Interpretation.
\end{remark}

\begin{remark}
Die Identitätsabbildung ist das neutrale Element der Abbildungsalgebra,
wie die $1$ das neutrale Element der Multiplikation ist ... daher auch $1_A$.
\end{remark}

\begin{remark}
Der Kompositionsoperator $∘$ verküpft zwei Abbildungen $g$ und $h$ zu 
einer neuen Abbildungen $h∘g$ ($h$ nach $g$).
\end{remark}

\chapter{Die Kategorie der Mengen}

Ein Beispiel für eine Kategorie:
Die Kategorie der endlichen Mengen, und Abbildungen.
Eine Abbildung \engl{Abbildung}{map} ist eine Prozess 
um von einer Menge in eine andere zu kommen.

% Nach Meijer also unsere erste Programmiersprache
% endliche Mengen als Typen
% Abbildungen als Funktionen
\begin{definition}[Kategorie der endlichen Mengen]
Die Kategorie \category{SET} besteht aus:
\begin{itemize}
\item Objekten: Mengen $A, B, \dots$
\item Morphismen: Abbildungen zwischen Mengen $f: A → B$ welche bestehen aus:
	\begin{itemize}
	\item Definitionsmenge: eine Menge $A$ \engl{Definitionsmenge}{domain}
	\item Zielmenge: eine Menge $B$ \engl{Zielmenge}{codomain}
	\item Abbildungsvorschrift: Zu jedem $a ∈ A$ ein $b ∈ B$ mit $f(a) = b$.
	\end{itemize}
\end{itemize}
\end{definition}

\begin{remark}
Die Abbildungen sind totale Funktionen (linkstotal und rechtseindeutig).
\end{remark}

% ------------------------------------------------------------------------------
% interne und externe diagramme 
\newpage
\subsection{Diagramme}

\begin{remark}[Interne Diagramme]
Der Inhalt finiter Mengen lässt sich durch (beschriftete) interne Diagramm 
darstellen:

\graphics{pics/internalDiagram1.png}{Internes Diagramm einer Menge}

Auch Abbildungen lassen sich auf diesem Weg darstellen;
die Abbildungsvorschrift kann durch interne Diagramme explizit gemacht werden:

\graphics{pics/internalDiagram2.png}{Internes Diagramm einer Abbildung}
\end{remark}

\begin{remark}[Externe Diagramme]
Externe Diagramme zeigen nur Definitions- und Zielmenge; 
sie werden verwendet wenn es auf die genaue Abbildungsvorschrift für den 
aktuellen Diskurs nicht relevant ist:

\graphics{pics/externalDiagram1.png}{Externe Diagramme}

Externe Diagramme können in jeder Kategorie verwendet werden,
da sie Morphismen zwischen Kategorien allgemein erfassen.
\end{remark}

% ------------------------------------------------------------------------------
\newpage
\subsection{Spezielle Abbildungen}

\begin{definition}[Endomorphismus]
Ein Endmorphismus $f: A → A$ ist ein Morphismus bei dem Quelle und Ziel 
identisch sind.
\end{definition}

\begin{definition}[Identitätsmprphismus]
Ein spezieller Endomorphismus ist der Identitätsmorphismus $1_A$,
der $A$ unverändert auf sich selbst abbildet. 
\end{definition}

\begin{remark}
In \category{SET} bildet der Identitätsmorphismus jedes Element auf sich selbst ab.
\end{remark}

\begin{remark}
Die Verknüpfungsabbildung zweier Morphismen $f$ und $g$ in \category{SET}
ist die Komposition $g∘f$ der Abbildungen;
Voraussetzung für die Komposition ist, dass die Zielmenge von $f$ identisch
mit der Definitionsmenge von $g$ ist.
\end{remark}

\graphics{pics/functionComposition1.png}{Funktionskomposition}

\begin{definition}[Punkt]% Unit?
Ein Punkt (auch Singleton) ist eine Menge mit exakt einem Element.
\end{definition}

\begin{remark}
Einelementige Mengen werden auch mit $1$ bezeichnet.
\end{remark}

\begin{theorem}
Die Komposition eines Punktes mit einer anderen Abbildung ist wieder ein Punkt.
\end{theorem}
\begin{proof}
% TODO
\end{proof}


%%%%%%%%%%%%%%%%%%%%%%%%%%%%%%%%%%%%%%%%%%%%%%%%%%%%%%%%%%%%%%%%%%%%%%%%%%%%%%%%
%% Session 2: Mengen, Abbildungen und Komposition                             %%
%%%%%%%%%%%%%%%%%%%%%%%%%%%%%%%%%%%%%%%%%%%%%%%%%%%%%%%%%%%%%%%%%%%%%%%%%%%%%%%%
\newpage

\begin{remark}[Vergleich/Identität von Abbildungen]

Die Abbildungsvorschrift ist der Inhalt, die Abbildung ist der Umfang.

\begin{example}
$$f : ℕ → ℕ, x ↦ (x+1)²$$
$$g : ℕ → ℕ, x ↦ x²+2x+1$$

Beide Abbildungen sind wertverlaufsgleich --- es handelt sich um dieselbe
Abbildung $f=g$. Das ist exakt die Aussage der folgenden Gleichung:

$$ (x+1)² = x²+2x+1$$

Im Allgemeinen drücken Gleichungen aus, das Ausdrücke mit (evtl. verschiedenen)
Inhalten den gleichen Umfang haben.
Der Begriff der Abbildung wird damit extensional verwendet,
da eine Abbildung von ihrem Umfang bestimmt wird, nicht intensional von  
ihrere Abbildungsvorschrift.
\end{example}
\end{remark}

\begin{remark}[Notwendige Bedingung für die Identität zweier Abbildungen]
Eine notwendige, aber nicht hinreichende Bedingung für die Identität zweier 
Abbildungen $f$ und $g$ ist in jeder Kategorie, dass $f$ und $g$ die gleichen 
Domänen und Kodomänen haben.
\end{remark}

\begin{remark}[Test der Identität zweier Abbildungen]
Um die Identität zweier Abbildungen $A \xrightarrow{f} B$ und $A \xrightarrow{g} B$
zu beweisen genügt es zu zeigen:
$$\text{Wenn für jeden Punkt} 1 \xrightarrow{a} A, f ∘ a = g ∘ a, \text{dann} f=g$$
\end{remark}

% Vielleicht als Beispiel: Fahrenheit nach Celsius S. 28

\begin{remark}[Unterschied zwischen Multiplikation und Komposition]
Multiplikation auf $ℕ$ ist kommutativ, Komposition nicht.
\end{remark}


\begin{remark}[Anzahl der Abbildungen von einer Menge in eine andere Menge]
Für endliche Mengen $A$ und $B$ gibt es $|B|^{|A|}$ viele Abbildungen 
$f : A → B$.
\end{remark}


%%%%%%%%%%%%%%%%%%%%%%%%%%%%%%%%%%%%%%%%%%%%%%%%%%%%%%%%%%%%%%%%%%%%%%%%%%%%%%%%
%% Session 3: Komposition von Abbildungen und Zählen von Abbildungen		  %%
%%%%%%%%%%%%%%%%%%%%%%%%%%%%%%%%%%%%%%%%%%%%%%%%%%%%%%%%%%%%%%%%%%%%%%%%%%%%%%%%

\begin{remark}
Die Menge der Abbildungen $f : A → B$ wird mit $B^A$ bezeichnet
(Da sie $|B|^{|A|}$ viele Elemente enthält).
\end{remark}


%%%%%%%%%%%%%%%%%%%%%%%%%%%%%%%%%%%%%%%%%%%%%%%%%%%%%%%%%%%%%%%%%%%%%%%%%%%%%%%%
%%%%%%%%%%%%%%%%%%%%%%%%%%%%%%%%%%%%%%%%%%%%%%%%%%%%%%%%%%%%%%%%%%%%%%%%%%%%%%%%
\setpartpreamble{%
\begin{abstract}
Wir untersuchen folgende Analogie: Wenn die Komposition von Abbildungen
der Multiplikation von Zahlen gleicht, was ist dann das Analogon zur Division?
\end{abstract}
}
\part{Die Algebra der Komposition}


%%%%%%%%%%%%%%%%%%%%%%%%%%%%%%%%%%%%%%%%%%%%%%%%%%%%%%%%%%%%%%%%%%%%%%%%%%%%%%%%
%% Artikel 2: Isomorphismen, 
%%%%%%%%%%%%%%%%%%%%%%%%%%%%%%%%%%%%%%%%%%%%%%%%%%%%%%%%%%%%%%%%%%%%%%%%%%%%%%%%
\chapter{Isomorphismen, Sektionen, Retraktionen, Idempotenz, Automorphismen}

% ------------------------------------------------------------------------------
\section{Isomorphie}

\begin{example}[Einführung: Vergleich von Mengen]
Isomorphismen erlauben unter abderem den Vergleich von Mengen ohne einen Zahlbegriff.
Ein Isomorphismus eine spezielle Abbildung. Welche Eigenschaften muss eine 
Abbildung $f: A → B$ haben, damit man zwei mit ihrer Hilfe zwei Mengen $A$ und 
$B$ vergleichen kann? Sie muss 
\begin{enumerate*}
\item injektiv und 
\item surjektiv 
\end{enumerate*}
sein --- eine Bijektion. Dann gibt es eine Umkehroperation $f^{-1}: B → A$ 
mit $f ∘ f^{-1} = 1_A$ und $f^{-1} ∘ f = 1_B$.
Beide dieser Gleichungen müssen erfüllt sein, damit $A$ und $B$ gleich mächtig
sind.
\end{example}

\graphics{pics/setIsomorphism1}{Isomorphismus zwischen Mengen}
% ein isomorphismus ist nur zwischen Mengen mit der gleichen Kardinalität möglich.
% ist die Anzahl der Isomorphismen zwischen zwei Mengen gleich ihrer Kardinalität?

\begin{definition}[Isomorphismus]\label{def:IsmomorphismAsInvertibleMorphism}
Ein Morphismus $f: A → B$ wird \definiendum{Isomorphismus} 
(oder umkehrbare Abbildung) genannt, wenn es eine Abbildung $g: B → A$ 
gibt für welche die folgenden Gleichungen gelten:
\begin{enumerate}
\item $f ∘ g = 1_A$
\item $g ∘ f = 1_B$
\end{enumerate}
Eine Abbildung $g$ welche in Bezug auf $f$ diese Eigenschaften erfüllt heißt 
\concept{Inverse von $f$} und wird auch mit $f^{-1}$ bezeichnet. Zwei Objekte 
$A$ und $B$ werden isomorph genannt, wenn es mindestens einen Isomorphismus 
$f: A → B$ gibt.
\end{definition}
% Beispiel 1:
% Descartes Koordinaten --- Isomorphismus zwischen Raum und der Algebra von n-Tupeln
% Beispiel 2:
% könnte sich jetzt überlegen wie der Curry-Howard-Lambek-Isomorphimus aussieht ...

% ------------------------------------------------------------------------------
% Isomorphie ies eine Äquivalenzrelation
\newpage

\begin{theorem}[Isomorphie ist eine Äquivlanzrelation]\label{thm:IsomorphismIsAnEquivalenceRelation}
Isomorphie zwischen zwei Objekten ist eine Äquivalenzrelation. Es gilt:
\begin{enumerate}
\item \label{def:IsoRef} Reflexivität: 
	$A$ ist isomorph zu $A$
\item \label{def:IsoSym} Symmetrie: 
	Wenn $A$ isomorph ist zu $B$, 
	dann ist $B$ isomorph zu $A$.
\item \label{def:IsoTra} Transitivität: 
	Wenn $A$ isomorph zu $B$, 
	und $B$ isomorph zu $C$,
	dann ist $A$ isomorph zu $C$.
\end{enumerate}
\end{theorem}

\begin{proof}
Seien $A$, $B$ und $C$ Objekte und $f: A → B$, $f^{-1}: B → A$ sowie $g: B → C$ 
und $g^{-1}: C → B$ Isomorphismen (mit ihren Umkehrabbildungen).
Dann gilt:
\begin{subproof}[Reflexivität]
Aufgrund der Identitätsabbildung $1_A$.
\end{subproof}
\begin{subproof}[Symmetrie] 
Aufgund der Voraussetzung.
\end{subproof}
\begin{subproof}[Transitivität]
\begin{align*}\label{sp:IsoTrans}
	  &f ∘ g ∘ g^{-1} ∘ f^{-1}  \\	% Assoziativität
	= &f ∘ 1_B ∘ f^{-1}			\\% Annahme
	= &f ∘ f^{-1}				\\% Identitätsgesetz
	= &1_A						% Annahme
\end{align*}
Weiter $ g ∘ f ∘ f^{-1}∘ g^{-1} = 1_C$ analog.
\end{subproof}
\end{proof}


\begin{theorem}[Eindeutigkeit des Umkehrmorphismus]\label{InvertibleMorphismIsUnique}
Hat ein Morphismus $f: A → B$ einen Umkehrmorphismus $f^{-1}: B → A$ ist dieser 
eindeutig.
\end{theorem}
\begin{proof}
%TODO
\end{proof}



% ------------------------------------------------------------------------------
% ------------------------------------------------------------------------------
\section{Entscheidungsprobleme: Wahl- und Determinationsproblem}
% section and retraction

In der Abbildungsalgebra gibt es zwei Arten von Entscheidungsproblemen:

\begin{definition}[Wahlprobleme (Lifting)]
\graphics{pics/liftingProblem.png}{Wahlproblem}
Gegeben ein $g$ und $h$, was sind alle $f$, so es welche gibt,
dass $g ∘ f = h$?
\begin{analogon}[Logik]
Beim Beweis einer Transitivität fehlt die erste Implikation.
\end{analogon}
\begin{analogon}[Multiplikation auf $ℕ$]
Achtung: Multiplikation ist kommutativ daher hinkt der Vergleich ...
Hinterer Faktor unbekannt: $g * ? = h$. 
\end{analogon}
\end{definition}

\begin{remark}
Das Wahlproblem entspricht dem Finden eines Morphismus' $f$ der auf die 
Repräsentanten gemäß $h$ von $C$ in $B$ abbildet.
\end{remark}

\begin{definition}[Determinationsproblem]
\graphics{pics/determinationsProblem.png}{Determinationsproblem}
Gegeben ein $f$ und ein $h$, was sind alle $g$ so es welche gibt,
dass $g ∘ f = h$?
\begin{analogon}[Logik]
Zum Beweis einer Transitivität fehlt die zweite Implikation.
\end{analogon}
\begin{analogon}[Multiplikation auf $N$]
Der vordere Faktor unbekannt: $? * g = h$.
\end{analogon}
\end{definition}

\begin{remark}
Das Determinationsproblem entspricht dem Finden eines Morphismus' $g$ der 
gemäß $h$ von den Repräsentaten von $C$ in $B$ auf $B$ abbildet.
\end{remark}

% ------------------------------------------------------------------------------
% Spezielle Wahlprobleme: Sektionsproblem und Retraktionsproblem

\subsection{Lösungen des spezielle Wahl- und Determinationsproblems: Sektion und Retraktion}

\begin{definition}[Spezielles Wahlproblem / Sektion]
\graphics{pics/sektionsProblem.png}{Sektionsproblem}
Beim spezielle Wahlproblemen gilt $h = 1_B$.
Lösungen dieses Problems sind Morphismen $s$ für die gilt $f ∘ s = 1_B$. 
Morphismen mit dieser Eigenschaft werden als Sektion von $f$ bezeichnet.
\end{definition}

\begin{definition}[Retraktion]
\graphics{pics/retraktionsProblem.png}{Retraktionsproblem}
Beim speziellen Determinationsproblem gilt $h = 1_A$.
Lösungen dieses Problems sind Morphismen $r$ für die gilt $r ∘ f = 1_A$.
Morphismen mit dieser Eigenschaft werden als Retraktionen von $f$ bezeichnet.
\end{definition}

\begin{remark}
Sektionen und Retraktionen sind weder eindeutig, noch existieren sie für jeden 
Morphismus. wie die folgenden Sätze belegen.
\end{remark}

\begin{theorem}[Nicht jede Abbildung hat eine Sektion]\label{thm:SektionNotNecessary}
Eine Abbildung $f$ hat nicht notwendig eine Sektion $s$.
\end{theorem}
\begin{proof}
% TODO
\end{proof}

\begin{theorem}[Nicht jede Abbildung hat eine Retrakion]\label{thm:RetractionNotNecessary}
Eine Abbildung $f: A → B$ hat nicht notwendig eine Retraktion 
\end{theorem}
\begin{proof}
\end{proof}

\begin{theorem}\label{thm:SectionNotUnique}
Eine Abbildung $f$ kann mehrere Sektionen haben.
\end{theorem}
\begin{proof}
% TODO
\end{proof}

\begin{theorem}\label{thm:RetractionNotUnique}
Eine Abbildung $f$ kann mehrere Retraktionen haben.
\end{theorem}
\begin{proof}
Analog \ref{thm:SectionNotUnique}.
\end{proof}

\begin{remark}
Manche Abbildungen haben Sektionen aber keine Retraktionen, manche Retraktionen
aber keine Sektionen, viele haben nichts von beidem.
\end{remark}

% ------------------------------------------------------------------------------
% Mit Sektionen lassen sich immer Lösungen des allgemeinen Wahlproblems konstruieren
\newpage
\subsection{Sektionen zur Lösung das allgemeine Wahlproblems}

\begin{remark}
Existiert zu einer Abbildung $f$ eine Sektion $s$ lassen sich damit immer 
Lösungen des allgemeinen Wahlproblems konstruieren, wie der nächste Satz belegt.
\end{remark}

\begin{theorem}[Sektionen lösen das allgemeine Wahlproblem]\label{thm:IfSectionLiftgingSolution}
\graphics{pics/sektionsProblem2.png}{Lösung des allgemeinen Wahlproblems}
Wenn eine Abbildung $f: B → A$ eine Sektion hat,
dann gibt es für jedes $T$ und für alle Abbildungen $y: T → B$ eine Abbildung
$x: T → B$ für die gilt: $f ∘ x = y$.
\end{theorem}
\begin{proof}
% TODO
\end{proof}

\begin{remark}
Theorem \ref{thm:IfSectionLiftingSolution} besagt, dass wenn das Sektionsproblem
als spezielles Entscheidungsproblem eine Lösung hat, dann haben alle 
allgemeineren strukturgleichen Entscheidungsprobleme eine Lösung.

\doublegraphics%
	{pics/sektionsProblem.png}{Spezielles Entscheidungsproblem}%
	{pics/sektionsProblem2.png}{Allgemeines Entscheidungsproblem}
\end{remark}

\begin{analogon}[Reziprokale]
Sei $x = \frac{1}{3}$ das Reziprokal der Gleichung $3 * s = 1$,
dann kann $3 * x = 5$ gelöst werden durch $x = s * 5 = \frac{1}{3} * 5$.
\end{analogon}

% das ist doch eher eine Definition oder?
\begin{theorem}[$f$ ist surjektiv für Abbildungen aus $T$]
Eine Abbildung $f$ ist surjektive für eine Abbildung aus $T$ wenn gilt:
für jedes $y$ gibt es ein $x$, so dass $f ∘ x = y$.
\end{theorem}
\begin{proof}
%TODO
\end{proof}

\begin{remark}[Sektionen als Zuweisungen von Repräsentaten]
Sektionen können auch als Zuweisungen von Repräsentaten in $A$ zu den Elementen
von $B$ verstanden werden.

\graphics{pics/sektionAlsRepraesentant.png}{Sektion als Repräsentant}
\end{remark}

% ------------------------------------------------------------------------------
% Mit Retraktionen sich Lösungen für das allgemeine Determinationsproblem konstruieren
\newpage
\subsection{Retraktionen zur Lösungen des allgemeine Determinationsproblems}

\begin{theorem}\label{thm:IfRestrictionDeterminationSolution}
Wenn eine Abbildung $f: A → B$ eine Retraktion hat,
dann gibt es für jede Abbildung $y: A → T$ eine Abbildung $t: B → T$ mit 
$t ∘ f = y$.
\end{theorem}

\begin{remark}
Theorem \ref{thm:IfRestricionAllSolution} besagt, dass wenn das Retraktionsproblem
als spezielles Determinationsproblem eine Lösung hat, dann haben alle 
allgemeineren strukturgleichen Determinationsprobleme eine Lösung.

\doublegraphics%
	{pics/retraktionsProblem.png}{Spezielles Determinationsproblem}%
	{pics/retraktionsProblem2.png}{Allgemeines Determinationsproblem}
\end{remark}

% ------------------------------------------------------------------------------
% Streichungseigenschaft 1, Monomorphismus
\newpage
\subsection{Eindeutigkeit, Injektivität, Monomorphismus}

\begin{definition}[Eindeutigkeitseigenschaft I]
Eine Abbildung $f: A → B$ besitzt die Eindeutigkeitseigenschaft I für ein 
Objekt $T$ und je zwei Morphismen $t₁: T → A$ und $t₂: T → A$ gilt:
$$\text{wenn\ } f ∘ t₁ = f ∘ t₂ \text{\ dann\ } x₁ = x₂$$

\graphics{pics/cancellationLeft.png}{Eindeutigkeit I}
\end{definition}

\begin{definition}[$f$ ist injektiv von $T$]
Wenn für eine Abbildung $f$ und ein $T$ gilt, 
die Eindeutigkeitseigenschaft I gilt nennt man $f$
\definiendum{injektiv für Abbildungen von $T$}.
\end{definition}

\begin{definition}[Monomorphimus]
Gilt für beliebeige $T$ und jede $t₁: T → A$ und $t₂: T → A$,
Eindeutigkeitseigenschaft I ist $f$ ein Monomorphismus (injektive Abbildung).
\end{definition}

\begin{theorem}[Retraktion impliziert Injektivität]\label{th:RetractionImpliesCancellation1}
Angenommen $f: A → B$ hat eine Retraktion.
Dann gilt je zwei Morphismen $x₁: T → A$ und $x₂: T → A$ von einem beliebigen 
Objekt $T$ die Streichungseigenschaft I, und $f$ ist ein Monomorphismus:

\graphics{pics/retraktionsEindeutigkeit.png}{Eindeutigkeit bei Retraktion}
\end{theorem}
\begin{proof}
%TODO
\end{proof}


% ------------------------------------------------------------------------------
% Streichungseigenschaft II und Epimorphismus
\newpage
\subsection{Eindeutigkeit II, Surjektivität, Epimorphismus}

\begin{definition}[Eindeutigkeitseigenschaft II]
Eine Abbildung $f: A → B$ besitzt die Eindeutigkeitseigenschaft II, wenn gilt:
$$\text{wenn\ } t₁ ∘ f = t₂ ∘ f \text{\ dann\ } t₁=t₂$$

\graphics{pics/cancellationRight.png}{Eindeutigkeit II}
\end{definition}

\begin{theorem}
Angenommen $f: A → B$ hat eine Sektion. Dann gilt für jedes Objekt $T$ und 
zwei Abbildungen $t₁: B → T$ und $t₂: B → T$, die Eindeutigkeitseigenschaft II:
$$\text{wenn\ } t₁ ∘ f = t₂ ∘ f \text{\ dann\ } t₁=t₂$$

\graphics{pics/sektionsEindeutigkeit.png}{Eindeutigkeit bei Sektion}
\end{theorem}
\begin{proof}
%TODO
\end{proof}

\begin{definition}[Epimorphismus]
Ein Morphismus $f$ mit dieser Streichungseigenschaft 
(wenn $t₁ ∘ f = t₂ ∘ f$ dann $t₁ = t₂$) für beliebige $T$ wird 
\definiendum{Epimorphismus} genannt.
\end{definition}


% ------------------------------------------------------------------------------
% Sektionen und Retraktionen sind abgeschlossen unter Komposition
\newpage
\subsection{Sektionen und Retraktionen sind abgeschlossen unter Komposition}

\begin{theorem}[Retraktionen sind abgeschlossen unter Komposition]
Wenn $f: A → B$ eine Retraktion hat, und $g: B → C$ eine Retraktion hat,
dann hat $g ∘ f: A → C$ eine Retraktion.
\end{theorem}
\begin{proof}
%TODO
\end{proof}

\begin{theorem}[Sektionen sind abgeschlossen unter Komposition]
Wenn $f: A → B$ eine Sektion hat, und $g: B → C$ eine Sektion hat,
dann hat $g ∘ f: A → C$ eine Sektion.
\end{theorem}
\begin{proof}
% TODO
\end{proof}

% ------------------------------------------------------------------------------
\newpage
\subsection{Idempotenz}

\begin{definition}[Idempotenz]
Ein Endomorphismus heißt \definiendum{idempotent}, wenn $e ∘ e = e$.
\end{definition}


% ------------------------------------------------------------------------------
% Isomorphismen lassen sich auch über Sektionen und Retraktionen definieren

\newpage
\subsection{Isomorphismus definiert über Sektion und Retraktion}

\begin{theorem}[Gibt es Sektion und Retrakion sind diese identisch]\label{thm:SectionAndRetractionIdentical}
Wenn $f: A → B$ sowohl eine Sektion $s$ als auch eine Retraktion $r$ hat,
dann $r = s$.
\end{theorem}
\begin{proof}
\end{proof}

\begin{definition}[Isomorphismus]
Ein Morphismus $f$ wird \definiendum{Isomorphismus} genannt, 
wenn es einen Morphismus $f^{-1}$ gibt, der sowohl Retraktion als auch Sektion
für $f$ ist.
\begin{align*}
f ∘ f^{-1} = 1_B \\
f^{-1} ∘ f = 1_A
\end{align*}
Der Morphismus $f^{-1}$ wir dann \concept{Inverse} von $f$ genannt;
die Inverse ist, so sie existiert eindeutig nach \ref{thm:SectionAndRetracionIdentical}.
\end{definition}


\begin{theorem}[Isomorphismen sind abgeschlossen unter Komposition]
Wenn $f: A → B$ und $g: B → C$ Isomorphismen sind, 
dann auch $g ∘ f: A → C$ mit der Inversen $(g ∘ f)^{-1} = f^{-1} ∘ g^{-1}$.
\end{theorem}
\begin{proof}
\end{proof}

\begin{theorem}[Isomorphismen erfüllen beide Eindeutigkeitseigenschaften]
Wenn $f$ einen inversen Morphismus hat, dann gelten die folgenden 
\informal{Eindeutigkeitseigenschaften}: %cancellation vll. auch Annulierung
\begin{enumerate}
\item Wenn $f ∘ g = f ∘ h$, dann $g = h$.
\item Wenn $g ∘ f = h ∘ k$, dann $g = h$.
\end{enumerate}
\end{theorem}
\begin{proof}
%TODO
\end{proof}

\begin{remark}[Bedeutung von Isomorphismen]
Was bedeutet es wenn zwei Objekte isomorph sind?
In der Kategorie \category{SET} bedeutet es, dass sie die gleiche Kardinalität
haben --- durch Isomorphie von Mengen wird ein Konzept der Mächtigkeit definiertm
dass ohne einen Zahlbegriff auskommt.
Andere Kategorien enthalten in der Regel Objekte mit mehr Struktur;
hier bedeutet Isomorphie, dass die Struktur der Objekte bei der Abbildung 
erhalten bleibt.
\end{remark}

% ------------------------------------------------------------------------------
% Endomorphismus & Isomorphismus: Automorphismus
\newpage
\subsection{Automorphismus}

\begin{definition}[Automorphismus]
Ein Endomorphismus $f: A → A$ der auch ein Isomorphismus ist,
wird \definiendum{Automorphismus} genannt.
\end{definition}

\begin{theorem}
Wenn es Isomorphismen $f: A → B$ gibt, ist ihre Anzahl gleich mit der 
Anzahl der Automorhpismen $a: A → A$.
\end{theorem}
\begin{proof}
\end{proof}

\begin{remark}
Ein Automorphismus in \category{SET} wird traditionell \concept{Permutation}
genannt.
\end{remark}

% Hier: Kategorie der Permutationen

\end{document}


