\documentclass{scrartcl}

\usepackage{common}

\begin{document}

\section{Verbände}

\subsection{Algebraische Beschreibung von Verbänden}

Algebraische Struktur: Menge mit inneren Verknüpfungen.

Inhalt:
\begin{itemize}
\item Definition/Axiomenmenge/Theorie der Verbände
\item Unterstrukturen
\item Strukturerhaltende Abbildungen (Homomorphismen)
\end{itemize}


\subsubsection{Definition von Verbänden als algebraische Struktur}

\begin{definition}{Verband (algebraische Struktur)}
Eine algebraische Struktur %$(V,⊔,⊓)$ 
\begin{itemize}
\item $V$: Trägermenge $V ≠ ∅$
\item $⊔: V × V → V$
\item $⊓: V × V → V$
\end{itemize}
welche die folgenden Axiome (Theorie der Verbände) erfüllt wird Verband genannt:
\begin{enumerate}
\item \axiom{Assoziativität}: 
    \begin{enumerate}
    \item $∀ x, y, z ∈ V: x ⊔ (y ⊔ z) = (x ⊔ y) ⊔ z$
    \item $∀ x, y, z ∈ V: x ⊓ (y ⊓ z) = (x ⊓ y) ⊓ z$
    \end{enumerate}
\item \axiom{Kommutativität}:
    \begin{enumerate}
    \item $∀ x, y ∈ V: x ⊔ y = y ⊔ x$
    \item $∀ x, y ∈ V: x ⊓ y = y ⊓ x$
    \end{enumerate}
\item \axiom{Absorption}:
    \begin{enumerate}
    \item $∀ x, y ∈ V: (x ⊓ y) ⊔ x = x$ 
    \item $∀ x, y ∈ V: (x ⊔ y) ⊔ x = x$ 
    \end{enumerate}
\end{enumerate}
\end{definition}

\begin{remark}[Alternative Bezeichner für die Verbandsoperationen]
Statt $⊔$, $⊓$ werden auch die Bezichner $∨$, $∧$ verwendet.
Motivation: Verband der Wahrheitswerte.
\end{remark}

\begin{remark}[Wichtige Verbände]
Wichtige Verbände:
\begin{itemize}
\item Verband der Wahrheitswerte: $(𝔹, ∨, ∧)$
\item Potenzmengenverband: $(2^X, ⋃, ⋂)$
\item Teilbarkeitsverband: $(ℕ, kgV, ggT)$
\end{itemize}
\end{remark}


\subsubsection{Sätze der Verbandstheorie}

\begin{theorem}[Idempotenz]
Sei $V = (V, ⊔, ⊓)$ ein Verband. Dann gilt \axiom{Idempotenz}: 
\begin{enumerate}
\item $∀ x ∈ V: x ⊓ x = x$
\item $∀ x ∈ V: x ⊓ x = x$
\end{enumerate}
\end{theorem}

\begin{proof}
\begin{align*}
    a ⊔ a 
    &= a ⊔ (a ⊓ (a ⊔ a)) && \text{\axiom{Absorption}} \\
    &= (a ⊓ (a ⊔ a)) ⊔ a && \text{\axiom{Kommutativität}} \\
    &= a
\end{align*}
\end{proof}


\begin{theorem}[FlipFlop]
Sei $V = (V, ⊔, ⊓)$ ein Verband. Dann gilt der \informal{FlipFlopsatz}:
$a ⊓ b = a ⇔ a ⊔ b = b$
\end{theorem}

\begin{proof}
\begin{subproof}[$a ⊓ b = a ⇒ a ⊔ b = b$]
\begin{align*}
    (a ⊓ b) ⊔ b = b \\
    &= a ⊔ b = b \\
\end{align*}
\end{subproof}
\begin{subproof}[$a ⊔ b = b ⇒ a ⊓ b = a$]
Dualisierung 
\end{subproof}
\end{proof}


\begin{theorem}[Gleichheitssatz]
Sei $V = (V, ⊔, ⊓)$ ein Verband. Dann gilt der \informal{Gleichheitssatz}:
$a ⊓ b = a ⊔ b ⇔ a = b$
\end{theorem}

\begin{proof}
\begin{subproof}[$a ⊓ b = a ⊔ b ⇒ a = b$]
\begin{align*}
    (a ⊔ b) ⊓ a &= a && \axiom{Idempotenz} \\
    (a ⊓ b) ⊓ a &= a && \text{Prämisse} \\
    (b ⊓ a) ⊓ a &= a && \axiom{Kommutativität} \\
    b ⊓ (a ⊓ a) &= a && \axiom{Kommutativität} \\
    b ⊓ a &= a && \axiom{Idempotenz} \\
    (b ⊓ a) ⊔ b &= b && \axiom{Idempotenz} \\
    a ⊔ b &= b && \text{Gleichung} \\
    a &= b && \text{Prämisse, Gleichung}
\end{align*}
\end{subproof}
\begin{subproof}[$a = b ⇒ a ⊓ b = a ⊔ b$]
\begin{align*}
    a &= b && \text{Prämisse} \\
    a ⊓ a &= b && \text{Idempotenz} \\
    a ⊓ a &= b ⊔ b && \text{Idempotenz} \\
    a ⊓ b &= a ⊔ b && \text{Prämisse}
\end{align*}
\end{subproof}
\end{proof}


\subsection{Unterstrukturen von Verbänden}
% Homomorphismen sind polymorph im Sinn von Overloading

\begin{definition}{Unterverband}
Sei $V$ ein Verband, $W ⊆ V$ mit $W ≠ ∅$.
Gilt für alle $a, b ∈ W$ auch $a ⊔ b ∈ W$ und $a ⊓ b ∈ W$ 
(\concept{Abgeschlossenheit bezüglich $⊔$ und $⊓$}) 
so nennt man $W$ sowie das Tripel 
$$(W, ⊔_{|W × W}, ⊓_{|W × W})$$
einen \concept{Unterverband} (bzw. \concept{Teilverband}) von $V$.
\end{definition}

\begin{remark}[Restriktionen]
Die Symbole $⊔_{|W × W}$ und $⊓_{|W × W}$ bezeichnen die Restriktionen von 
$⊔$ und $⊓$ auf $W$; liegt keine Ambiguität vor können zur Denotation der
Restriktionen auch die Symbole $⊔$ und $⊓$ verwendet werden.
\end{remark}

\begin{remark}[Untermengen sind nicht notwendig abgeschlossen]
Nicht jede nicht-leere Untermenge eines Verbandes induziert einen Verband;
\property{Abgeschlossenheit} muss nicht gelten.
\end{remark}

\begin{proof}
Sei $V = (2^{\set{1,2}}, ∪, ∩)$, $W = \set{\set{1},\set{2}}$ dann ist 
$ \set{1} ∪ \set{2} ∉ W$.
\end{proof}


\subsection{Verhandshomomorphismen}

\begin{definition}{Verbandshomomorphismus}
Seien $V$, $W$ Verbände. Eine Abbildung $f: V → W$ nennt man 
Verbandshomomorphismus, wenn gilt:
$$∀ a, b, c ∈ V: (f(a ⊔_V b) = f(a) ⊔_W f(b) ∧ f(a ⊓_V b) = f(a) ⊓_W f(b))$$
\end{definition}

\begin{definition}{Spezielle Verbandshomomorphismen}
Verbandshomomorphismen können weiter klassifiziert werden:
\begin{itemize}
\item Verbandsmonomorphismus: injektiver Verbandshomomorphismus
\item Verbandsepimorphismus: surjektiver Verbandshomomorphismus
\item Verbandsisomorphismus: bijektiver Verbandshomomorphismus
\end{itemize}
\end{definition}


\begin{theorem}{Verbandshomomorphismen sind abgeschlossen unter Komposition}
Seien $f$, $g$ Verbandshomomorphismen, dann auch $f∘g$ 
\end{theorem}

\begin{proof}
% Siehe Dunn & Hardegree ... sollte universelle Eigenschaft von Homomorphismen sein
% mit Verbandshomomorphismen als Spezialfall.
\end{proof}


\begin{theorem}{Verbandsisomorphismen sind abgeschlossen unter Inversenbildung}
Sie $f$ ein Verbandshomomorhpismus, dann auch $f^{-1}$
\end{theorem}

\begin{proof}
% TODO
\end{proof}


%-------------------------------------------------------------------------------
\section{Geordnete Mengen}

\begin{definition}{Ordnungsrelation}
Eine Relation $⊑ ⊆ M × M$ auf einer Menge $M ≠ ∅$ heißt Ordnungsrelation
wenn gilt:
\begin{itemize}
\item \axiom{Reflexivität}: $∀ x ∈ M: (x,x) ∈ ⊑$
\item \axiom{Antisymmetrie}: $∀ x, y ∈ M: (x,y) ∈ M ∧ (y,x) ∈ M → x = y$
\item \axiom{Transitivität}? $∀ x, y, z ∈ M: (x,y) ∈ M ∧ (y,z) ∈ M → (x,z) ∈ M$
\end{itemize}
\end{definition}
% Prototyp: Teilmengenbeziehung

\begin{definition}{Striktordnungsrelation}
Eine Relation $⊏ ⊆ M × M$ auf einer Menge $M ≠ ∅$ heißt Striktordnungsrelation
wenn gilt:
\begin{itemize}
\item \axiom{Irreflexivität}: $∀ x ∈ M: ¬((x,x) ∈ ⊏)$
\item \axiom{Transitivität}: $∀ x, y ,z ∈ M: (x,y) ∈ ⊏ ∧ (y,z) ∈ ⊏ → (x,z) ∈ ⊏$
\end{itemize}
\end{definition}

\begin{definition}{Strikter Anteil einer Ordnungsrelation}
Konstruktion einer Striktordnung aus einer gegebenen Ordnung:
Sei $⊑$ eine Ordnung. Die Relation $⊏$ mit
$a ⊏ b ⇔ a ⊑ b ∧ a ≠ b$ ist eine Striktordnungsrelation;
man nennt sie den \concept{strikten Anteil} von $⊑$.
\end{definition}

\begin{definition}{Reflexive Hülle einer Striktordnung}
Konstruktion einer Ordnung aus einer gegebenen Striktordnung:
Sei $⊏$ eine Striktordnung. Die Relation $⊑$ mit 
$a ⊑ b ⇔ a ⊏ b ∨ a = b$ ist eine Ordnungsrelation;
man nennt sie die \concept{reflexive Hülle} von $⊏$.
\end{definition}


\begin{definition}{Vergleichbarkeit (Kommensurabilität)}
Sei $(M, ⊑)$ eine Ordnung. Zwei Elemente $x, y ∈ M$ nennt man vergleichbar
(kommensurabel) gdw. $a ⊑ b$ oder $b ⊑ a$.
Zwei Elemente sind unvergleichbar (inkommensurabel) gdw. sie nicht vergleichbar
sind.
\end{definition}


\begin{definition}{Mengentheoretische Kette}
Sei $(M,⊑)$ eine Ordnung. Eine Teilmenge $K$ von $M$ heißt (mengentheoretische)
Kette, wenn gilt: $∀ a, b ∈ K: (a ⊑ b ∨ b ⊑ a)$
\end{definition}

\begin{definition}{Totalordnung}
Sei $(M,⊑)$ eine Ordnung. Ist $M$ eine Kette nennt man $M$ Totalordnung.
\end{definition}


\begin{definition}{Antikette}
Sei $(M,⊑)$ eine Ordnung. Eine Teilmenge $K$ von $M$ heißt Antikette,
wenn gilt: $∀ a, b ∈ K: a ⊑ b ↔ a = b$.
\end{definition}

% Ergänzend:
% Ketten mit Indexbereichen
% Quasiordnungen



\subsection{Unterstrukturen von Mengen}

\begin{definition}{Teilordnung}
Sei $(M, ⊑)$ ein Ordnung und sei $∅ ≠ N ⊆ M$.
Dann heißt $(N, ⊑_N)$ die durch $N$ induzierte Teilordnung.
\end{definition}

\begin{remark}[Restriktion]
Das Symbol $⊑_N$ bezeichnet die Restriktionen von $⊑$ auf $N$:
liegt keine Ambiguität vor kann zur Denotation der
Restriktionen auch das Symbol $⊑$ verwendet werden.
\end{remark}

\begin{remark}[Jede Restriktion einer Ordnung ist wieder eine Ordnung]
Jede Restriktion einer Ordnung ist wieder eine Ordnung.
\end{remark}


\subsection{Strukturerhaltende Abbildungen}

\begin{definition}{Ordnungshomomorphismus/monotone Abbildung}
% Wichtig: Ordnungshomomorphismus == monotone Abbildung
Seien $(M_1, ⊑_1)$ und $(M_2, ⊑_2)$ Ordnungen.
Eine Abbildung $f: M_1 → M_2$ nennt man Ordnungshomomorphismus 
(bzw. monotone Abbildung) wenn gilt:
$∀ a, b ∈ M_1: a ⊑_1 b → f(a) ⊑_2 f(b)$
\end{definition}

\begin{definition}{Dualisierung der Ordnung/antitone Abbildung}
Seien $(M_1, ⊑_1)$ und $(M_2, ⊑_2)$ Ordnungen.
Eine Abbildung $f: M_1 → M_2$ nennt man Ordnungshomomorphismus 
(bzw. antitone Abbildung) wenn gilt:
$∀ a, b ∈ M_1: a ⊑_1 b → f(a) ⊑_2 f(b)$
\end{definition}

\begin{definition}{Spezielle Ordnungshomomorphismen}
Ordnungshomomorphismen können weiter klassifiziert werden:
\begin{itemize}
\item Ordnungsisomorphismus: bijektiver Ordnungshomomorphismus dessen Umkehrabbildung
		ebenfalls ein Homomorphismus ist.
\end{itemize}
\end{definition}


\begin{theorem}{Ordnungshomomorphismen sind abgeschlossen unter Komposition}
Seien $f$, $g$ Ordnungshomomorphismen/monotone Abbildungen, dann auch $f∘g$ 
\end{theorem}

\begin{proof}
% TODO vermutlich wieder allg. Algebra
\end{proof}


\begin{theorem}{Ordnungsisomorphismen sind nicht abgeschlossen unter Inversenbildung}
Sie $f$ ein Verbandshomomorhpismus, dann ist $f^{-1}$ im Allgemeinen kein
Verbandshomomorphismus.
\end{theorem}

\begin{proof}
Betrachte die Ordnungen $(2^\set{1,2}, ⊆)$ und $(\set{0,1,2,3}, ≤)$,
sowie die monotone Abbildung 
$f = \set{(∅, 0), (\set{1}, 1), (\set{2}, 2), (\set{1,2}, 3)}$.
Dann ist $f^-1$ keine Monotone Abbildung da $1 ≤ 2$ aber $¬(\set{1} ⊆ \set{2})$.
\end{proof}

\begin{remark}{Isomorphismen in algebraischen und relationalen Strukturen}
In relationalen Strukturen sind bijektive Homomorphismen nicht notwendig 
Isomorphismen. Die gemeinsam mit der Bijektivität für einen Ordnungsisomorphismus 
hinreichende Verstärkung (Umkehrabbildung ebenfalls monoton) muss 
definitiorisch ergänzt werden.
\end{remark}


\subsection{Spezielle Elemente in Ordnungen und Teilmengen}

\begin{definition}{Schranken}
Sei $(M, ⊑)$ eine Ordnung, $N ⊆ M$ und $a ∈ M$.
Dann heißt das Element $a$:
\begin{itemize}
\item \concept{obere Schranke} von $N$, wenn gilt: $∀ b ∈ N: b ⊑ a$.
\item \concept{untere Schranke} von $N$, wenn gilt: $∀ b ∈ N: a ⊑ b$.
\end{itemize}
\end{definition}

\begin{definition}{Maximale/Minimale Elemente}
Sei $(M, ⊑)$ eine Ordnung, $N ⊆ M$ und $a ∈ M$.
Dann heißt das Element $a$:
\begin{itemize}
\item \concept{Maximales Element} von $N$, wenn gilt: $¬∃ b ∈ N: a ⊑ b ∧ a ∈ N$.
\item \concept{Minimales Element} von $N$, wenn gilt: $¬∃ b ∈ N: b ⊑ a ∧ a ∈ N$.
\end{itemize}
\end{definition}

\begin{definition}{Größtes/Kleinstes Element}
Sei $(M, ⊑)$ eine Ordnung, $N ⊆ M$ und $a ∈ M$.
Dann heißt das Element $a$:
\begin{itemize}
\item \concept{Größtes Element} von $N$, wenn gilt: $∀ b ∈ N: b ⊑ a ∧ a ∈ N$.
\item \concept{Kleinstes Element} von $N$, wenn gilt: $∀ b ∈ N: a ⊑ b ∧ a ∈ N$.
\end{itemize}
\end{definition}

\begin{remark}[Existenz spezieller Elemente ist nicht notwendig]
Die oben aufgeführten Elemente müssen nicht notwendeigerweise in jeder Ordnung
existieren.
\end{remark}

\begin{remark}[Eindeutigkeit spezieller Elemente]
Angenommen die oben aufgeführten Elemente existieren in beliebigen aber festen
Ordnung.
\begin{itemize}
\item Schranken und maximalen/minimalen Elementen sind in der Regel nicht eindeutig.
\item Größtes und kleinstes Element sind eindeutig.
\end{itemize}
\end{remark}


\begin{theorem}{Endliche nichtleere Ordnungen haben maximale und minimale Elemente}
Sei $(M, ⊑)$ eine Ordnung, dann besitzt jede endliche, nichtleere Teilmenge $N$
ein maximales und ein minmales Element.
% genau eins? vermutlich nicht ...
\end{theorem}

\begin{proof}[Konstruktiver/Algorithmischer Beweis]
Konstruktiver/algorithmischer Beweis: linerare Suche nach einem maximalen Element.

Betrachte die extensionale Darstellung von $N = \set{a_1, \dots , a_n}$ mit $n ≥ 1$.
Folgende Konstruktion definiert eine aufsteigende Folge
(deren Elemente eine Kette bilden) $x_1 ⊑ x_2 ⊑ \dots ⊑ x_n$:
\begin{align*}
    x_1 &= a_1 \\
    x_{i+1} &= 
        \begin{cases}
            a_{i+1} & \text{falls} x_i ⊏ a_{i+1} \\
            x_i     & \text{sonst}
        \end{cases}
\end{align*}

Der Beweis für das minimale Element verläuft analog.
\end{proof}



\subsection{Majorante und Minorante}

\begin{definition}{Ma und Mi}
Sei $(M, ⊑)$ eine Ordnung. Zwei wichtige Funktionen:
\begin{enumerate}
\item Ma (von \concept{Majorante} = obere Schranke) bildet 
		ein $N ⊆ M$ auf die Menge ihrer oberen Schranken ab.
\item Mi (von \concept{Minorante} = untere Schranke) bildet
		ein $N ⊆ M$ auf die Menge ihrer unteren Schranken ab.
\end{enumerate}
\begin{align*}
    &\mobj{Ma}: 2^M → 2^M \\
    &\mobj{Ma}(X) = \comp{a ∈ M}{a \text{ist obere Schranke von} X} \\
    &\mobj{Mi}: 2^M → 2^M \\
    &\mobj{Mi}(X) = \comp{a ∈ M}{a \text{ist untere Schranke von} X}
\end{align*}
\end{definition}


\begin{theorem}[Teilmengenbeziehung und Antitonizität]
Sei $(M, ⊑)$ ein Verband, $N_1, N_2 ∈ 2^M$, dann gilt:
Wenn $N_1 ⊆ N_2$ dann $Ma(N_2) ⊆ Ma(N_1)$ und $Mi(N_2) ⊆ Mi(N_1)$ --
\mobj{Ma} und \mobj{Mi} sind antitone Abbildungen.
\end{theorem}

\begin{proof}
Sei $N_1 ⊆ N_2$, dann gilt:
\begin{align*}
\mobj{Ma}(N_2) &= \comp{a ∈ M}{∀ b ∈ N_2: b ⊑ a} && \text{Definition \mobj{Ma}} \\
               &⊆ \comp{a ∈ M}{∀ b ∈ N_1: b ⊑ a} && N_1 ⊆ N_2 \\
               &= \mobj{Mi}(N_1) && \text{Definition \mobj{Mi}}
\end{align*}
Antitonie von \mobj{Mi} wird äquivalent gezeigt.
\end{proof}


\begin{theorem}{???}
Sei $(M, ⊑)$ ein Verband, $N ∈ 2^M$, dann gilt:
\begin{itemize}
\item $\mobj{Ma}(\mobj{Mi}(\mobj{Ma}(N))) = \mobj{Ma}(N)$
\item $\mobj{Mi}(\mobj{Ma}(\mobj{Mi}(N))) = \mobj{Mi}(N)$
\end{itemize}
\end{theorem}

\begin{proof}
% TODO
\end{proof}


\subsection{Supremum und Infimum}

% Supremum und Infimum sind sehr wichtig

\begin{definition}{Supremum und Infimum}
Sei $(M, ⊑)$ eine Ordnung, $N ⊆ M$ und $a ∈ M$. Dann heißt $a$:
\begin{itemize}
\item \concept{Infimum} von $N$, falls $a$ das größte Element von $\mobj{M}(N)$ ist.
\item \concept{Supremum} von $N$, falls $a$ das kleinste Element von $\mobj{Ma}(N)$ ist.
\end{itemize}
\end{definition}

\begin{remark}[Supremum und Infimum als partielle Abbildungen]
Wenn Suprema bzw. Infima existieren sind diese eindeutig.
Folgende partielle Abbildungen können definiert werden:
\begin{itemize}
\item $∏: 2^M → M$ für das Infimum (größte untere Schranke)
\item $∐: 2^M → M$ für das Supremum (kleinste obere Schranke)
\end{itemize}
\end{remark}

\begin{theorem}[Teilmengen und Supremum/Infimum]
Sei $(M,⊔,⊓)$ eine Ordnung, $N_1, N_2 ⊆ M$ und $N_1 ⊆ N_2$, dann gilt:
$∏N_2 ⊑ ∏N_1$ und $∏N_1 ⊑ ∏N_2$, falls diese Suprema und Infima existieren.
\end{theorem}

\begin{proof}
Angenommen $a, b ∈ M$ mit $a=∐N_1$ und $b=∏N_2$.
\begin{subproof}[$∐N_1 ⊑ ∏N_2$]
\begin{align*}
    b = ∐N_2 &⇒ b ∈ \mobj{Ma}(N_2) && \text{Definition Supremum} \\
             &⇒ b ∈ \mobj{Ma}(N_1) \\
             &⇒ a ⊑ b && \text{Definition \mobj{Definition Supremum}}
\end{align*}
\end{subproof}
\end{proof}




%-------------------------------------------------------------------------------
\section{Verbände als speziell geordnete Mengen}

Verbände stehen zu speziellen Ordnungen in einer eindeuting umkehrbaren
Beziehung; sie können folglich als algebraische sowie relationale Struktur 
definiert werden.

Zwei Konstruktionsprinzipen:
\begin{enumerate}
\item Konstruktion einer Ordnung aus einem gegebenen Verband
\item Konstruktion eines Verbands aus einer gegebenen Ordnung
\end{enumerate}


\subsection{Konstruktion einer Ordnung aus einem gegebenen Verband}

\begin{theorem}{Konstruktion einer Ordnung aus einem gegebenen Verband}
Sei $(V,⊔,⊓)$ ein Verband. Definiert man auf seiner Trägermenge $V$ eine 
Relation $⊑$ durch $a ⊑ b ⇔ a ⊓ b = a$ für alle $a, b ∈ V$, 
so ist $⊑$ eine Ordnungsrelation auf $V$.
\end{theorem}

\begin{proof}
\begin{subproof}[Reflexivität]
Sei $a ∈ V$ beliebig aber fest, so gilt:
\begin{align*}
    a ⊑ a &⇔ a ⊓ a 
          &⇔ a = a
\end{align*}
\end{subproof}
\begin{subproof}[Antisymmetrie]
Seien $a, b ∈ V$ beliebig aber fest, so gilt: 
\begin{align*}
    a ⊑ b ∧ b ⊑ a &⇒ a ⊓ b = a ∧ b ⊓ a = b \\
                  &⇒ a = a ⊓ b = b
\end{align*}
\end{subproof}
\begin{subproof}[Transitivität]
Seien $a, b, c ∈ V$ beliebig gewählt, so gilt:
\begin{align*}
    a ⊑ b ∧ b ⊑ c &⇔ a ⊓ b = a ∧ b ⊓ c = b \\
                  &⇒ a ⊓ b ⊓ c = a \\
                  &⇒ a ⊓ c = a 
\end{align*}
\end{subproof}
\end{proof}


\begin{theorem}
Für alle Teilmengen der Form $\set{a,b} ⊆ V$ gelten die folgenden Gleichungen:
\begin{align*}
    ∐\set{a,b} = a ⊔ b \\
    ∏\set{a,b} = a ⊓ b
\end{align*}
\end{theorem}

\begin{proof}
% TODO
\end{proof}



\subsection{Konstruktion eines Verbands aus einer gegebenen Ordnung}

\begin{theorem}{Konstruktion eines Verbands aus einer gegeben Ordnung}
Sei $(M,⊑)$ eine Ordnung mit der Eigenschaft, das für $a, b ∈ M$ sowohl
$∐\set{a,b}$ als auch $∏\set{a,b}$ existieren.
Definiert man zwei Abbildungen:
\begin{enumerate}
\item $⊔: M × M → M$, $a ⊔ b = ∐\set{a,b}$ 
\item $⊓: M × M → M$, $a ⊓ b = ∏\set{a,b}$
\end{enumerate}
dann ist $(M,⊔,⊓)$ ein Verband.
\end{theorem}

\begin{proof}
\begin{subproof}[Assoziativität]
\begin{align*}
    (a ⊔ b) u c &= ∐\set{∐\set{a,b},c} \\
                &= ∐\set{a,b,c} \\
                &= ∐\set{a,∐\set{b,c}} \\
                &= a ⊔ (b ⊔ c)
\end{align*}
Für $⊓$ analog.
\end{subproof}
\begin{subproof}[Kommutativität]
\begin{align*}
    a ⊔ b &= ∐\set{a,b} \\
          &= ∐\set{b,a} = b ⊔ a
\end{align*}
Für $⊓$ analog.
\end{subproof}
\begin{subproof}[Idempotenz]
Es gilt $a ⊑ ∐\set{a,b} = a ⊔ b$. Daraus folgt:
\begin{align*}
    (a ⊔ b) ⊓ a &= ∏ (a ⊔ b, a) \\
                &= a
\end{align*}
\end{subproof}
Für das zweite Absorptionsgesetz analog.
\end{proof}


\begin{theorem}[Jeder Verbandshomomorphismen ist ein Ordnungshomomorphismen]
Seien $V_1, ⊔_1, ⊓_2)$ und $(V_2, ⊔_2, ⊓_2)$ Verbände, $f: V_1 → V_2$ ein
Verbandshomomorphismus, so ist $f$ monoton bezüglich der induzierten
Ordnungen $⊑_1$ und $⊑_2$ (also ein Ordnungshomomorphismus).
\end{theorem}

\begin{proof}
\begin{align*}
    a ⊑_1 b &⇔ a ⊓_1 b = a \\
            &⇒ f(a ⊓_1 b) = f(a) \\
            &⇒ f(a) ⊓_2 f(b) = f(a) \\
            &⇒ f(a) ⊑_2 f(b)
\end{align*}
\end{proof}

\begin{theorem}[Nicht jeder Ordnungshomomorphismus ist ein Verbandshomomorphismus]
\end{theorem}

\begin{proof}
% TODO
\end{proof}

\begin{remark}[Keine Extensionsgleichheit von Ordnungs- und Verbandshomomorphismen]
In Verbänden denotieren die Begriffe Ordnungs- und Verbandshomomorphismus nicht 
die gleiche Abbildungsmenge: Die Menge der Ordnungshomomorphismen ist eine 
Obermenge der Menge der Verbandshomomorphismen.
\end{remark}

\begin{theorem}[Abschätzungen]
Seien $(V_1, ⊔_1, ⊓_1)$ und $(V_2, ⊔_2, ⊓_2)$ Verbände, $f: V_1 → V_2$ ein
Ordnungshomomorphsimus bezüglich der induzierten Verbandsordnungen $⊑_1$ und $⊑_2$ 
so gelten für alle $a, b ∈ V_1$ folgende Abschätzungen:
\begin{itemize}
\item $f(a ⊓_1 b) ⊑_2 f(a) ⊓_2 f(b)$
\item $f(a ⊔_1 b) ⊒_2 f(a) ⊔_2 f(b)$
\end{itemize}
\end{theorem}

\begin{proof}
Sei $V$ ein Verband, $f$ eine monotone Abbildung und seien $a, b ∈ V$ beliebig aber fest.
\begin{subproof}[$f(a ⊓_1 b) ⊑_2 f(a) ⊓_2 f(b)$]
\begin{align*}
    a ⊓_1 b &= ∏_1\set{a,b} \\
    ⇒ ∏_1\set{a,b} ⊑_1 a\\
    ⇒ a ⊓_1 b ⊑_1 a \\
    ⇒ f(a ⊓_1 b) ⊑_2 f(a)
\end{align*}
\begin{align*}
    a ⊓_1 b &= ∏_1\set{a,b} \\
    ⇒ ∏_1\set{a,b} ⊑_1 b\\
    ⇒ a ⊓_1 b ⊑_1 a \\
    ⇒ f(a ⊓_1 b) ⊑_2 f(b)
\end{align*}
Daraus folgt $f(a ⊓_1 b) ⊑_2 ∏_2\set{f(a),f(b)} = f(a) ⊓_2 f(b)$. % warum folgt das?
\end{subproof}
\begin{subproof}
Analog.
\end{subproof}
\end{proof}

\begin{theorem}[Verbandsisomorphismen und Ordnungsisomorphismen sind extensionsgleich]
Seien $(V_1, ⊔_1, ⊓_1)$ und $(V_1, ⊔_1, ⊓_1)$ Verbände und $f: V_1 → V_2$, dann gilt:
$f$ ist ein Verbandsisomorphismus gdw. $f$ ein Ordnungsisomorphismus bzgl.
der induzierten Ordnungen $⊑_1$ und $⊑_2$ ist.
\end{theorem}

\begin{proof}
\begin{subproof}[$⇒$]
% TODO
\end{subproof}
\begin{subproof}
% Todo
\end{subproof}
\end{proof}

\end{document}

%\subsection{Dualitätsprinzip der Verbandstheorie}


\subsection{Nachbarschaft und Diagramme}


\begin{definition}[Nachbar]
Sei $V$ ein Verband und $a ∈ V$.
\begin{itemize}
\item Ein Element $u ∈ V$ heißt \concept{unterer Nachbar} von $a$ falls gilt:
	$u ⊏ a ∧ ∀ b ∈ V: u ⊑ b ⊑ a → (u=b ∨ a=b)$
\item Ein Element $o ∈ V$ heißt \concept{oberer Nachbar} von $a$ fall gilt:
	$a ⊏ o ∧ ∀ b ∈ V: a ⊑ b ⊑ o → (o=b ∨ a=b)$
\end{itemize}
\end{definition}


\begin{definition}[Atom]
Sei $V$ ein Verband mit kleinstem Element $O ∈ V$.
Jeder obere Nachber von $O$ wird \concept{Atom} genannt.
$\mobj{At}(V)$ bezeichnet die \concept{Menge aller Atome} von $V$.
\end{definition}


\begin{definition}[Atomarer Verband]
Sei $V$ ein Verband mit kleinstem Element $O ∈ V$.
$V$ heißt atomarer Verband, falls es zu jedem $b ∈ V$ mit $b ≠ O$ ein 
Atom $a ∈ \mobj{At}(V)$ mit $a ⊑ b$ gibt.
\end{definition}


\begin{comment}
% ich denke nicht, dass das für die Arbeit besonders relevant ist
\begin{theorem}[Jeder endliche Verband ist atomar]
Jeder endliche Verband $V$ besitzt ein kleinstes Element und ist atomar.
\end{theorem}

\begin{proof}
\begin{subproof}[Jeder endliche Verband besitzt ein kleinstes Element]
Jeder endliche Verband besitzt eine extensionale Darstellung;
das kleinste Element der in extensionaler Darstellung $\set{a_1, \dots, a_n}$
ist durch $a_1 ⊓ \dots ⊓ a_2$ gegeben.
\end{subproof}
\begin{subproof}[Jeder endliche Verband ist atomar]
\begin{case}[$|V| = 1$]
Trivial.
\end{case}
\begin{case}[$|V| ≥ 2$]
Beweis durch Widerspruch. Angenommen $V$ ist nicht atomar, 
dann ist $V$ nicht endlich:
\begin{induction}
\base{TODO}
\hypothesis{TODO}
\step{TODO}
\end{induction}
Widerspruch zur Annahme.
\end{case}
\end{subproof}
\end{proof}
\end{comment}


\begin{theorem}
Sei $(V, ⊔, ⊓)$ und $a, b ∈ \mobj{At}(V)$ mit $a ≠ b$, dann gilt: $a ⊓ b = O$.
\end{theorem}

\begin{proof}
Beweis durch Widerspruch.
Seien $a, b ∈ \mobj{At}(V)$. Angenommen es gilt $a ⊓ b ≠ O$, dann gilt:
\begin{align*}
      &a ⊓ b ⊑ a ∧ a ⊓ b ⊑ b
    ⇒ &a ⊓ b = a ∧ a ⊓ b = b
    ⇒ a = b
\end{align*}
Widerspruch zur Annahme.
\end{proof}


\begin{definition}[Dualer Verband]
Sei $(V, ⊔, ⊓)$ ein Verband. Vertauscht man $⊔$ mit $⊓$ erhält man den 
\concept{dualen Verband} $(V, ⊓, ⊔)$ mit invertierter (dualer) Ordnung.
\end{definition}



\begin{definition}[Hasse Diaramm]
Zu einer Ordnung $(M, ⊑)$ heißt die Relation $H_⊑$ der unteren Nachbarschaft
das \concept{Hasse-Diagramm}: $ a H_⊑ b ⇔ a \text{ist unterer Nachbar von} b$.

Gilt für alle $a, b ∈ M$ die Beziehung $a ⊑ b$ genau dann, wenn $a H_⊑^* b$ gilt,
mit $H_⊑^*$ als der reflexiv-transitiven Hülle von $H_⊑$ so besitzt die Ordnung
per Definition ein Hasse-Diagramm. %???
\end{definition}

\begin{theorem}[Jede endliche Ordnung besitzt ein Hasse-Diagramm]
Jede endliche Ordnung besitzt ein Hasse-Diagramm.
\end{theorem}

\begin{proof}
Beweislos.
\end{proof}

\begin{theorem}[Nicht jede Ordnung besitzt ein Hasse-Diagramm]
Nicht jede Ordnung besitzt ein Hasse-Diagramm.
\end{theorem}



\subsection{Einige Konstruktionsmechanismen}

\subsubsection{Produktverband}

\subsubsection{Abbildungsverband}

\subsubsection{Quotientenverband}

\subsubsection{Lifting}



% ------------------------------------------------------------------------------
\section{Wichtige Verbandsklassen}


% ------------------------------------------------------------------------------
\section{Fixpunkttheorie}
















