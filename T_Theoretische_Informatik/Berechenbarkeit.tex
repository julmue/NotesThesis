\documentclass{scrartcl}
\usepackage{common}
\begin{document}
%-------------------------------------------------------------------------------
%-------------------------------------------------------------------------------
%-------------------------------------------------------------------------------
\section{Berechenbarkeitstheorie}

Gegenstände der Berechenbarkeitstheorie sind:
\begin{enumerate*}
\item\label{it:Berechenbarkeitsbegriff} 
	Die Formalisierung des intuitiven Berechenbarkeitsbegriffs/Algorithmenbegriffs.
\item\label{it:Einteilung} 
	Einteilung der Funktionen anhand dieses Berechenbarkeitsbegriffs in 
	berechenbare (also algorithmisch lösbare) und nicht-berechenbare Funktionen.
\item\label{it:Entscheidbarkeit} 
	Einteilung der berechenbaren Funktionen in entscheidbare und 
	nicht-entscheidbare Funktionen.
\end{enumerate*}

% Sind Entscheidbare Probleme immer total? Stimmt diese Klasse mit den Totalen
% Funktionen überein?
% Vermutlich nicht ... es kann ja auch totale zufällige Funktionen geben.
% Aber vielleicht mit den totalen berechenbaren?

Die in \ref{it:Berechenbarkeitsbegriff} gestellte Frage ist historisch von 
größter Bedeutung --- ihre Lösung war der erste Schritt zur Lösung des von 
Hilbert gestellten Entscheidungsproblems. 
Dabei wurde sie durch mehrere Ansätze gelöst:
\begin{enumerate*}
	\item Turing-Maschinen
	\item Lambda-Kalkül
	\item Rekursive Funktionen
	\item Goto-Berechenbarkeit
	\item Loop-Berechenbarkeit
\end{enumerate}
Diese Ansätze bedeuten alle die gleiche Klasse von Funktionen. %siehe Church-Turing-These
Der zweite Schritt zur Lösung des Entscheidungsproblems war es das 

Der dritte Schritt zur Lösung des Entscheidungsproblems.

Wichtige Ergebnisse:
\begin{description}
	\item [Die Church-Turing-These I]
			Der von Turing-maschinen festgelegt Berechenbarkeitsbegriff
			erfasst genau die Menge der berechenbaren Funktionen. 
	\item [Die Church-Turing-These II]
			Die oben genannten Berechenbarkeitsbegriffe beschreiben die gleiche
			Menge von Funktionen:
	\item [Es gibt berechenbare und nicht-berechenbare Funktionen] 
			Nicht jede Funktion ist berechenbar/algorithmisierbar:
			Entscheidungskriterium für die algorithmische Unlösbarkeit 
			bestimmter Probleme.
	\item [Es gibt entscheidbare und nicht-entscheidbare Probleme]
			Nicht jedes Problem für dass es eine berechenbare Funktion gibt ist
			entscheidbar. %das gegebproblem muss auch entscheidbar sein.
\end{description}

%-------------------------------------------------------------------------------
%-------------------------------------------------------------------------------
\subsection{Intuitiver Berechenbarkeitsbegriff}

Zunächst die Beschreibung des intuitiven Berechenbarkeitsbegriffs nach 
\autocite{TIKG:2003}:

\begin{definition}[Intuitiver Berechenbarkeitsbegriff]
Eine (evtl. partielle) Funktion $f : Def(f) → ℕ$ soll als berechenbar angesehen 
werden, falls es einen Algorithmus gibt, der $f$ berechnet:
\begin{itemize}
\item totale Funktion $Def(f) = ℕ^k$:
	\begin{description}
	\item [Input] 
		$(n₁,\dots,n_k) ∈ ℕ^k$ 
	\item [Output] 
		Termination mit $f(n₁,\dots,n_k)$ nach endlich vielen Schritten.
	\end{itemize}
\item partielle Funktionen $Def(f) ⊆ ℕ^k$ modelliert durch Funktion $f'$:
	\begin{itemize}
	\item [Input] $(n₁,\dots,n_k) ∈ ℕ^k$ 
	\item [Output] 
		\begin{align*}
		f'(n₁,\dots,n_k) = 
		\begin{cases}
			\text{terminiert mit} f(n₁,\dots,n_k), \text{falls} (n₁,\dots,n_k) ∈ Def(f) \\
			⊥, \text{sonst}
		\end{align*}
	\end{itemize}
\end{itemize}
mit $Def(f) ∩ ⊥ = ∅$; $⊥$ bezeichnet Definitionslücken.
\end{definition}

\begin{remark}
Für alle Eingabewerte gilt: 
Wenn der Eingabewert keine Definitionslücke der Funktion ist
dann muss die Berechnung auch in endlicher Zeit terminieren;
Ist der Eingabewert eine Definitionlücke muss der Algorithmus nicht 
terminieren.
\end{remark}

\begin{remark}
Dieser Berechenbarkeitsbegriff ist nicht konstruktiv!
Es genügt der Nachweis, dass ein Algorithmus existiert, dieser muss
nicht explizit angegeben werden.

\begin{example}
Folgende Funktion ist berechenbar:
\begin{align*}
	n_{i} &= 
		\begin{cases}
    	    1 & \text{falls das LBA-Problem eine positive Lösung hat}
    	    0     & \text{sonst}
    	\end{cases}
\end{align*}
\end{example}
Die epistemische Lücke ob der Lösbarkeit des LBA-Problems ändert nichts 
daran, dass $i$ berechenbar ist.
\end{remark}

\begin{remark}
Jedem Algorithmus ist also eine (partielle) Funktion zugeordnet;
ein Algorithmus ist eine finite, intensionale Darstellung einer 
(u.U. transfiniten) (u.U. partiellen) extensionalen Funktion;
Jeder Algorithmus denotiert/bezeichnet/bedeutet eine Funktion.
\end{remark}

\begin{remark}
Zwei unterschiedliche Algorithmen können die gleiche Funktion bedeuten.
Sie haben dann einen unterschiedlichen Inhalt (Sinn) aber den gleichen
Umfang (Bedeutung).

Es gibt im Allgemeinen keinen Algorithmus, der gegeben zwei Algorithmen mit 
unterschiedlichem Inhalt feststellen kann ob diese die gleiche Bedeutung haben.
(Lambdakalkül, Halteproblem).
\end{remark}

% \begin{remark}[Berechenbarkeit von Zahlen]
% Die Funktionen der Zahlen $π$ und $e$ sind berechenbar.% ??? sind doch nur Näherungsverfahren
% Nicht jede reele Zahl ist berechenbar.
% Die Kardinalität der berechenbaren Funktionen ist gleich der Kardinalität von $ℕ$,
% und damit kleiner als die Kardinalität von $ℝ$.
% 
% Jede natürliche Zahl ist berechenbar, jede rationale Zahl ist berechenbar.
% \end{remark}


%-------------------------------------------------------------------------------
%-------------------------------------------------------------------------------
\subsection{Formaler Berechenbarkeitsbegriff}

\begin{theorem}[Churchsche These]
Die durch die formale Definition der \concept{Turing-Berechenbarkeit}
(äquivalent: \concept{Lambda-Definierbarkeit}, \concept{While-Berechenbarkeit},
\concept{Goto-Berechenbarkeit}, μ-Rekursivität, \dots) erfasste Klasse von
Funktionen stimmt genau mit der Klasse der im intuitiven Sinn berechenbaren
Funktionen überein.
\end{theorem}

\subsubsection{Turing-Berechenbarkeit}

\begin{remark}
% Bislang wurde die Turing-Maschinen zum Akzeptieren von Sprachen verwendet.
% Nun muss die Definition modifiziert werden, um das Berechnen von Funktionen
% zu erfassen. 
Im folgenden Abschnitt werden zwei Definitionen von Turing-Berechenbarkeit 
vorgestellt, eine für Funktionen auf natürlichen Zahlen, eine für Funktionen 
auf Wörtern\autocite[87]{TIKG:2003}.
\end{remark}

\begin{definition}Turing-Berechenbarkeit für Funktionen über $ℕ$}\label{def:TuringNat}
Eine Funktion $f: ℕ^k → ℕ$ heißt \concept{Turing-Berechenbar},
falls es eine (deterministische) Turingmaschine $M$ gibt,
so dass für alle $n₁,\dots,n_k$ gilt:
\begin{align*}
	f(n₁,\dots,n_k) &= m \\
				&gdw. \\
	z₀bin(n1)#bin(n₂)#\dots#bin(n_k) &⊢^* ⎕\dots⎕z_ebin(m)⎕\dots⎕
\end{align*}
wobei $z_e ∈ E$.
\end{definition}

\begin{definition}{Turing-Berechenbarkeit für Funktionen über Wörtern}\label{def:TuringWord}
Eine Funktion $f : Σ^* → Σ^*$ heißt \concept{Turing-berechenbar},
falls es eine (deterministische) Turingmaschine $M$ gibt,
so dass für alle $x,y ∈ Σ^*$ gilt:
\begin{align*}
	f(x) &= y
		&gdw.
	z₀x ⊢^* ⎕\dots⎕z_ey⎕\dots⎕
\end{align*}
wobei $z_e ∈ E$.
\end{definition}

\begin{remark}[Turing-Maschinen und Nicht-Termination]
In Definition \ref{def:TuringWord} und Definition \ref{def:TuringNat} implizit
enthalten ist, dass im Falle von $f(x) = ⊥$ die Maschine in eine 
unendliche Schleife gehen kann. 
\begin{remark}

\begin{remark}[Turing-Maschinen und Typ-0-Sprachen]
Die Turingmaschine muss nur für Eingaben halten die durch die Funktion definiert 
sind. Für nicht definierte Eingaben muss die Maschine nicht halten.

Dies betrifft auch das Wortproblem von Typ-0-Sprachen;
diese entsprechen damit exakt den semi-entscheidbaren Sprachen\autocite{TIKG:2003}.
\end{remark}

% \begin{theorem}
% Zu jeder Mehrband-Turingmaschine $M$ gibt es eine Einband-Turingmaschine $M'$
% mit $T(M) = T(M')$ bzw. so, dass $M'$ dieselbe Funktion berechnet wie $M$.
% \end{theorem}
% T ist hier die von der Turingmaschine akzeptierte Sprache


%%-------------------------------------------------------------------------------
%\subsubsection{Loop Berechenbarkeit}
%
%\mobj{Loop} ist eine einfache Programmiersprache.
%\fo{Loop}-Programme sind aus den folgenden Komponenten aufgebaut:
%\begin{itemize}
%\item Variablen: $x₁, x₂, x₃, \dots$
%\item Konstanten: \loop{0 1 2} \dots
%\item Trennsymbole: \loop{; :=}
%\item Operationszeichen: \loop{+ -}
%\item Schlüsselwörter: \loop{LOOP DO END}
%\end{itemize}
%
%\begin{definition}[Syntax von \fo{Loop}-Programmen]
%\begin{align*}
%	L &= x_i \loop{:=} x_j + c
%	  &= x_i \loop{:=} x_j - c
%	  &| L₁\loop{;}L₂)
%	  &| \Loop{LOOP}x₁\loop{DO}P\loop{END}
%\end{align*}
%\end{definition}
%
%\begin{definition}[(Operationale) Semantik von \fo{Loop} Programmen]
%%TODO
%\end{definition}
%
%\begin{theorem}
%Es gilt: Wenn eine Funktion \fo{Loop}-berechenbar ist,
%dann ist sie total.
%\end{theorem}
%
%\begin{remark}
%Mit \fo{Loop}-Programmen ist es nicht möglich eine unendliche Schleife zu bauen;
%jede Schleife terminiert nach einiger Zeit.
%\end{remark}
%
%\begin{example}
%Additionsfunktion.
%\end{example}
%
%\begin{question}
%Entspricht Loop einem bestimmten Autoamtenkonzept?
%wohl nicht...
%\end{question}
%
%\begin{theorem}
%Es gilt: Wenn eine Funktion total ist, 
%dann ist sie nicht notwendig \fo{Loop}-berechenbar.
%\end{theorem}
%\begin{proof}
%Ackermann-Funktion.
%\end{proof}
%
%\begin{theorem}
%Eine Funktion ist \fo{Loop}-berechenbar gdw. sie primitiv-rekursiv ist.
%\end{theorem}
%
%
%%-------------------------------------------------------------------------------
%\subsubsection{While Berechenbarkeit}
%
%\mobj{While} ist eine einfache Programmiersprache.
%\fo{While}-Programme sind aus den folgenden Komponenten aufgebaut:
%\begin{itemize}
%\item Variablen: $x₁, x₂, x₃, \dots$
%\item Konstanten: \while{0 1 2} \dots
%\item Trennsymbole: \while{; :=}
%\item Operationszeichen: \while{+ -}
%\item Schlüsselwort: \while{LOOP DO END}
%\item Schlüsselwort: \while{WHILE}$x₁ ≠ 0$ \while{DO} $p$ \while{END}
%\end{itemize}
%
%\begin{definition}[Syntax von While-Programmen]
%\end{definition}
%
%\begin{definition}[Semantik von While-Programmen]
%\end{definition}
%
%\begin{remark}
%Loop und While sind keine historisch bedeutsamen Programmiersprachen,
%sondern nur zu didaktischen Zwecken von Schöning entworfen
%\end{remark}
%


%-------------------------------------------------------------------------------
%-------------------------------------------------------------------------------
\subsection{Primitiv-rekursive Funktionen}

Die primitiv-rekursiven Funktionen sind eine wichtige echte Untermenge der 
totalen, berechenbaren Funktionen. Sie werden durch Komposition und
durch primitive Rekursion aus atomaren Grundfunktionen gebildet:

\begin{definition}[primitiv-rekursive Funktionen]
Die Klasse der \concept{primitiv-rekursiven Funktionen} auf
den natürlichen Zahlen):
\begin{enumerate}
\item\label{it:basis1} Alle konstanten Funktionen sind primitiv-rekursiv.
\item Alle identischen Funktionen (Projektionen) sind primitiv rekursiv.
	% (Beispiel $f : A × B × C → B$ mit $f(a,b,c) ↦ b)$)
\item\label{it:basis2} Die Nachfolgerfunktion $succ(n) = n + 1$ auf $ℕ$ ist primitiv rekursiv.
\item\label{it:basis3} Primitiv-rekursive Funktionen sind abgeschlossen unter Komposition.
\item Primitiv-rekursive Funktionen sind abgeschlossen unter primitive Rekursion:
	Seien $f, g$ primitiv rekursive Funktionen, $f$ eine primitiv-rekursive Funktion
	die durch primitive Rekursion aus $f$ und $g$ gebildet wird.
	Dann $f$ muss ein Gleichungssystem der folgenden Art erfüllen:
	\begin{align*}
		f(0,\dots) &= g(\dots)
		f(n+1,\dots) &= h(f(n,\dots),\dots)$
	\end{align*}
\end{itemize}
Die Funktionen \ref{it:basis1}-\ref{it:basis3} werden Basisfunktionen genannt.
\end{definition}

\begin{example}[Additionsfunktion $add: ℕ × ℕ → ℕ$]
Die Additionsfunktion $add: ℕ × ℕ → ℕ$ ist primitiv-rekursiv:
\begin{align*}
	add(0,x) &= x
	add(n + 1, x) &= succ(add(n,x))
\end{align*}
\end{example}

\begin{theorem}
Für alle berechenbaren Funktionen $f$ gilt: 
Wenn $f$ primitiv rekursiv ist, dann ist $f$ total.
\end{theorem}

\begin{proof}
% TODO
\end{proof}

\begin{theorem}
Für alle berechenbaren Funktionen $f$ gilt:
Wenn $f$ total ist, dann ist $f$ nicht notwendig primitiv rekursiv.
\end{theorem}

\begin{proof}
Die Ackermann-Funktion ist eine totale, intuitiv berechenbare Funktion 
die nicht primitiv-rekursiv ist.
\end{proof}

% \begin{question}
% Welche Rolle spielen die primitiv rekursiven Funktionen in der Beweistheorie
% und im Curry Howard isomorphismus?
% ist es möglich Termination zu kapseln, beispielsweise in einer Monade
% \end{question}

%-------------------------------------------------------------------------------
%-------------------------------------------------------------------------------
\subsection{$μ$-rekursive Funktionen}

Eine echte Erweiterung der primitiv-rekursiven Funktionen wird durch
Hinzunahme des \comcept{$μ$-Operators} erreicht. 
Die daraus entstehende Klasse von Funktionen wird 
\concept{$μ$-rekursive Funktionen} oder \concept{partiell-rekursive Funktionen}
genannt. 
Die Klasse der $μ$-rekursiven Funktionen ist extensionsgleich mit der durch 
Turing-Berechenbarkeit festgelegten Klasse der berechenbare Funktionen;
es sind also auch partielle Funktionen enthalten.

\begin{definition}
Sei $f: ℕ^{k + 1} → ℕ$ eine gegebene (evtl. partielle) Funktion.
Die durch Anwendung des $μ$-Operators auf $f$ entstehende Funktion ist 
$μf = g$ mit $g: ℕ^k → ℕ$ und
$ g(x₁,\dots,x_k) = 
	\text{min} \{ n | f(n,x₁,\dots,x_k) = 0, ∀ m<n: f(m,x₁,\dots,x_k) \text{ist definiert}\}$
mit $min ∅ = ⊥$.
\end{definition}
%Das sieht doch gaaaaanz extrem nach dem least-fixpoint aus?

\begin{remark}
Durch Anwendung des $μ$-Operators können partielle Funktionen entstehen.
\end{remark}

\begin{definition}
Die Klasse der $μ$-rekursiven Funktionen ist die kleinste Klasse von
(evtl. partiellen) Funktionen, di die Basisfunktionen enthält und 
abgeschlossen ist unter Einsetzung, primitiver Rekursion und 
Anwendung des $μ$-Operators.
\end{definition}

\begin{theorem}
Die Klasse der $μ$-rekursiven Funktionen stimmt genau mit der Klasse 
der Turing-berechenbaren Funktionen überein.
\end{theorem}

% \begin{question}
% Wie ist der gute Kurt eigentlich auf diese Konzept gekommen?
% Erscheint mir doch iwi reichlich unintuitiv ...
% \end{question}


%-------------------------------------------------------------------------------
%-------------------------------------------------------------------------------
\subsection{Entscheidbarkeit}

\begin{definition}[Entscheidbarkeit]
Eine Menge $A ⊆ Σ^*$ heißt \concept{entscheidbar},
falls die \concept{charakteristische Funktion} von $A$,
$χ_A: Σ^* → \set{0,1}$ berechenbar ist.
Es gilt für alle $w ∈ Σ^*$:
\begin{align*}
	χ_A(w) = 
	\begin{cases}
	1, w ∈ A \\
	0, w ∉ A
	\end{cases}
\end{align*}
\end{definition}

\begin{definition}[Semi-Entscheidbarkeit]
Eine Menge $A ⊆ Σ^*$ heißt \concept{semi-entscheidbar},
falls die \enquote{halbe} \concept{charakteristische Funktion} von $A$,
$χ_A: Σ^* → \set{0,1}$ berechenbar ist.
Es gilt für alle $w ∈ Σ^*$:
\begin{align*}
	χ'_A(w) =
	\begin{cases}
	1, w ∈ A \\
	⊥, w ∉ A
	\end{cases}
\end{align*}
\end{definition}

% Es liegt keine Entscheidungsprozedur vor ...
% im Falle 

\begin{remark}
Diese Definitionen lassen sich auch auf Mengen $A ⊆ ℕ^k$ übertragen. 
\end{remark}

\begin{remark}
Im Zusammenhang mit der Frage der Entscheidbarkeit werden Sprachen auch oft 
(Entscheidungs-) Probleme genannt.
\end{remark}

\begin{theorem}
Eine Sprache $A$ ist entscheidbar gdw. sowohl $A$ als auch $\bar{A}$ 
semi-entscheidbar sind.
\end{theorem}

\begin{theorem}
Eine Sprache ist rekursiv aufzählbar, gdw. sie semi-entscheidbar ist.
\end{theorem}

\begin{example}[SAT]
\end{example}


%-------------------------------------------------------------------------------
% Entscheidungsproblem
% Quelle: Wikipedia
\end{document}
